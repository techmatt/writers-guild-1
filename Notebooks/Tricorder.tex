%%%%%
%%
%% Research Notebooks live in this directory.  This file doubles as a
%% latex'able example notebook.
%%
%% Notebook macros (in ../Lists/notebook-LIST.tex, presumably) each
%% have a file that lives here.  The argument to \startnotebook{...}
%% probably should be the macro for the given whitesheet.  However,
%% you can also just use \name{Some Text} if you want.
%%
%% Note that every \startnotebook command needs a matching
%% \endnotebook command.  Also note that no ownership information
%% appears on the notebook.
%%
%%%%%

\documentclass[notebook]{elementals}
\begin{document}

\startnotebook{\nTricorder{}}

\begin{page}{first}

In the years since elementals were introduced to this plane, humans have spent a lot of time and energy developing technologies to address this threat. This \iTricorder{} has the same basic functionality as all \iTricorder{} - namely detecting elemental strength. This feature is often used to determine whether humans choose to engage an elemental or retreat.

If you wish to detect the strength of an elemental, spend 30 seconds investigating with the \iTricorder{}. The player must then reveal their $\beta$ score to you. $\beta$ scores mean:
\begin{enumerate}
  \item ``0'' Not an elemental.
  \item ``1'' Elemental of low rank.
  \item ``2'' Elemental of middling rank.
  \item ``3'' Highest ranking elemental known to humans.
  \item ``4+'' Elementals of this strength have never been observed on Earth.
\end{enumerate}

If you would like to program the \iTricorder{} to detect if there is elemental residue on something or someone, turn to page \nbref{second}.

If you would like to program the \iTricorder{} to detect the location of closest contact, turn to page \nbref{fourth}.

\end{page}

\begin{page}{second}

Detecting the type of elemental residue on something can be crucial in certain circumstances, particularly in criminal investigations. Unfortunately it is not a standard issue functionality. If you would like to program the \iTricorder{} to do this, persuade 3 different elementals to recite the alphabet.

When you complete this step, you may turn to page \nbref{third}.

\end{page}

\begin{page}{third}

The \iTricorder{} can now detect elemental residue.

If you wish to detect what type of elemental residue is present on a person, item or location, spend 30 seconds investigating with the \iTricorder{}, then read the $\alpha$ score for the item, or request it from the person being examined. If the score is:

\begin{enumerate}
  \item a ``0'', then there is no elemental residue.
  \item a ``1'', then there is fire elemental residue.
  \item a ``2'', then there is water elemental residue.
  \item a ``3'', then there is earth elemental residue.
  \item a ``4'', then there is air elemental residue.
  \item a ``5'', then there is elemental residue of some kind, but it doesn't match any known elemental type.
\end{enumerate}

\end{page}

\begin{page}{fourth}

The \iTricorder{} was never meant to locate the point of closest contact, but you think that you can jury-rig it to do so.

In order to program the \iTricorder{}, {\bf First}, spend 1 minute discussing how to locate the conduit with someone who has successfully done so. {\bf Second}, spend 1 minute discussing with an elemental who has been to the elemental plane what it is like. {\bf Third}, spend 1 minute discussing with a scientist (\cGD{}, \cScientist{}, or \cMS{}) how to reprogram machines. (If you are a scientist, you may simply spend 1 minute in meditation on the question.)

When you complete these 3 steps, you may turn to page \nbref{fifth}.

\end{page}

\begin{page}{fifth}

The \iTricorder{} can now locate where the conduit can be opened from. While it is known that the point of closest contact is in this general vicinity, the exact location remains unknown. To further complicate matters, the point of closest contact may move throughout the evening as the planes approach each other.

The display on the \iTricorder{} now reads: {\bf {\large 487}}

In order to establish the point of closest contact, and thus the place where the conduit can currently be opened, you must follow these steps:
\begin{enumerate}
  \item Hold the \iTricorder{} in front of ``A Possible Location of Closest Contact'' Sign for 30 seconds.
  \item {\bf Only once step 1 has been completed}, lift the first page of the ``possible location'' sign up to read the chart on the second page. Locate the number associated with the current time. This is the active number.
  \item Check whether the number displayed on the machine is a factor of the active number on the \iTricorder{} display. If the number on the \iTricorder{} is a factor of the number on the sign, then this location is currently the point of closest contact and the conduit can be opened from this location - if you know how. (For example, if the machine display reads ``5'' and the active location number was ``20'' then that location would be the point of closest contact and the conduit {\bf could} be opened here. But if the active number was ``23'', then the location would not be the point of closest contact and the conduit {\bf could not} be opened here.) \emph{You can't check numbers that aren't currently active.} 
\end{enumerate}
\end{page}

\endnotebook

\end{document}

%
%\begin{page}{third}
%
%\end{page}
%
%\begin{page}{fourth}
%
%
%\end{page}
%
%\begin{page}{fifth}
%
