\documentclass[green]{elementals}
\begin{document}
\name{\gEnslave{}} 

Enslavement is a CR driven ability. The mechanic works like this:

\begin{enumerate}
 \item The user points at their target and states ``Enslave'' and their CR (do {\bf not} count assists or item modifiers. Any CR \emph{penalties} {\bf do} affect CR for this ability). You may ``pull your punch'' with this ability. 
 
 For example, if I have a base CR of 4 and the ability ``Enslavement (x),'' but am currently suffering from a CR -1 penalty, I would point at my target and say ``Enslave 3''. 
 
 \item The target player may resist this ability with their CR if they are able to do so (do {\bf not} count item modifiers. Any CR \emph{penalties} {\bf do} affect CR for resisting this ability). This is comparable to resisting any other darkwater CR based attack. If the target player does so, the ability fails. 
 
 For example, if my target has a CR of 3, they can simply say ``Resist'', and my ability fails.
 However, if my target is suffering from a CR -1 penalty, they \emph{cannot} resist because their effective CR is only 2.
 
 \item If the target player does {\bf not} immediately resist, the user of the ability must now incant ``Enslaving 1,'' ``Enslaving 2,'' etc, up to a 10-count (``Enslaving 10''). {\bf The target player is completely mesmerized and can take NO action during the count except to ``resist'' if they are able to and choose to do so.}
 
 \item Any {\bf other} player can interrupt the action during the count by saying ``I stop you'' from within 1 ZoC, or attacking the person using the ability.
 
 \item Once the 10-count has finished, the ability has been successfully used, so the target cannot resist, and no one can interrupt the action.
\end{enumerate}

If you are enslaved, you are compelled to do as your enslaver says for 5 minutes. You {\bf can} be compelled to perform any action that is not suicide. You {\bf can} be compelled  to discuss a given topic. You {\bf cannot} be compelled to reveal specific information unless you know otherwise. For example: ``Tell me about this machine'' is a valid command, but ``Tell me what you plan to do with this machine'' is not. Your enslaver may release you before the five minutes elapse by pointing at you and saying ``I release you''.

{\bf If you are successfully enslaved, you may freely resist further attempts at enslavement for 30 minutes following your release from enslavement. (Regardless of CR.)}

\end{document}
