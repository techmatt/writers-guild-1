\documentclass[green]{elementals}
\begin{document}
\name{\gConduit{}}

Read the entire sheet before beginning the process. A GM must be present before step 2 begins.


If you would like to open the conduit, you must follow these steps:
\begin{enumerate}
  \item Locate the current point of closest contact.
  \item Consume 1 battery(item number: \iBattery{\MYnumber{}}) (tear up the item card).
  \item Declare {\bf loudly} (but don't scare non-players) that you are opening the conduit. Someone around the corner should be able to hear you.
  \item Place your ability card ``\aWorkConduit{}'' on the ``\gConduitModifier{}'' sheet.
  \item {\bf If anyone other than the person who is using ``\aWorkConduit{}'' wishes to modify the opening of the Conduit, their ability cards must be placed on the ``\gConduitModifier{}'' sheet at this time as well. ALL ABILITIES MUST BE VISIBLE. Otherwise nothing will happen when the conduit is opened. Further, each person may only submit {\bf one} ability.}
  \item All players with abilities on ``\gConduitModifier{}'' spend 5 minutes opening the conduit.
  \item The GM will reveal what happens.
\end{enumerate}

Important notes:
\begin{itemize}
  \item Anyone who is not involved in opening the conduit or modifying the process in some way can walk up and observe the abilities being used. (\emph{The actions you are performing are obvious}).
  \item Any action taken by players involved in opening or modifying the conduit, including attacking/defending, or talking to players not involved in opening the conduit constitutes {\bf \emph{withdrawing}} their own action.
  \item Noticing an attempt at waylay constitutes {\bf \emph{withdrawing}} your action as well.
  \item A person involved in opening or modifying the conduit may {\bf \emph{withdraw}} at any time by removing their ability card from the ``\gConduitModifier{}'' sheet. However, if they do so, any effects of the ability will not occur.
  \item Opening the conduit as well as all modifications are intimately linked and {\bf interruptible}. This means that {\bf interrupting} \emph{any} of the abilities being used prevents \emph{all} actions from occurring. {\bf \emph{Withdrawing}} an action does not interrupt other actions unless it is the ``\aWorkConduit{}'' ability; in which case, all modifications fail to occur, because the conduit is no longer being opened.
\end{itemize}

\end{document}
