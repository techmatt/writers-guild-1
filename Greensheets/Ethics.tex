\documentclass[green]{elementals}
\begin{document}
\name{\gEthics{}}

Given the elemental incident 100 years ago, relatively tight protocols have been devised to prevent a similar disaster from occurring again. Much of what constitutes ``ethical'' science obeys extensions of the basic procedures established in the late 20th century in the aftermath of the Stanford Prison Experiment and the Milgram Experiment. While what constitutes acceptable is continuously debated from first principles and in light of the elemental threat, relatively strong precedents have been established:

\begin{enumerate}
\item All participants in scientific experiments should be willing and offered the right to refuse at any time during the procedure.
\item Experiments should never cause permanent psychological or physical changes to the participant (except in the very rare case where the participant fully understands and accepts this). Almost all re-taping of Azi falls under this condition, unless it is performed over a very long period of time requiring, at a minimum, several months.
\item All new procedures should be well documented so that others can understand them if something goes wrong. Destroying, falsifying, or encrypting the experimental procedure or resulting data is unacceptable.
\end{enumerate}

When there is ``reasonable suspicion'' that a scientist is going beyond the bounds of what is considered ``safe'' research, the official response is for a detailed investigation of their lab work to be conducted. As always, what exactly constitutes reasonable suspicion is subjective. Typically, if this investigation shows that dangerous research is being performed, the guilty scientist's person is searched, their license is revoked, and they are publicly denounced and watched.

Mechanically, a lab raid must be performed by a scientist and at least one other person (typically, a politician approving the search). The two investigators must search around the entire lab space for five minutes. This action is very obvious and can be easily interrupted. At the end of this time, you may interact with the appropriate sign for that lab, to get a summary of the research the scientist is conducting.

\end{document}
