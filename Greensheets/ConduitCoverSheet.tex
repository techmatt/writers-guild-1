\documentclass[green]{elementals}
\begin{document}
\name{\gConduitCover{}}

%%Cart goes Beep Beep

This green-sheet explains how to use this machine.

The machine is fairly complex and only a few functions are available without substantial study. If you have ability ``\aWorkConduit{}'', You may access the additional features on the green-sheet below this one. {\bf If you do not possess this ability, you may only interact with this machine via this green-sheet. In other words, you are limited to the interactions available on this sheet.}

{\bf Locate the Point of Closest Contact:}
This machine can locate where the conduit can be opened from. While it is known that the point of closest contact is in this general vicinity, the exact location remains unknown. To further complicate matters, the point of closest contact may move throughout the evening as the planes approach each other.

In order to establish the point of closest contact, and thus the place where the conduit can be opened at that time, you must follow these steps:
\begin{enumerate}
  \item Maneuver this machine to be directly in front of ``A Possible Location of Closest Contact'' Sign.
  \item Check whether the number on the sign matches the current number as indicated in the ``Possible Conduit Locations Notebook'' \emph{Please note that you cannot determine whether a location is valid unless you have completed step one at that location immediately prior.} If the numbers match, you have found the current location of Closest Contact between the planes.
\end{enumerate}

\end{document}
%%Machine has 1 number and instructions
%%Each conduit location has 2 pages. covered page has list of numbers that are active at different times
%%Check for prime(?) factors