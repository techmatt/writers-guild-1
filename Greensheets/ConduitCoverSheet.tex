\documentclass[green]{elementals}
\begin{document}
\name{\gConduitCover{}}

%%Cart goes Beep Beep

The machine is fairly complex and only one function is available without substantial study. If you have the ability ``\aWorkConduit{}'', You may access the additional features on the green-sheet below this one. {\bf If you do not possess this ability, you may only interact with this machine via this green-sheet. In other words, you may only use the machine to do do then things indicated by this greensheet.}

NOTE: This machine is big, heavy and unwieldy. It is 3 hands bulky (one person may move it by walking heel to toe, two may move it walking normally). If you move this machine, you must indicate that the machine is moving by saying ``Beep beep'' continuously. We wouldn't want to run someone over with it.


{\large Available Machine Functions:}


{\bf Locate the Point of Closest Contact:}
This machine can locate where the conduit can be opened from. While it is known that the point of closest contact is in this general vicinity, the exact location remains unknown. To further complicate matters, the point of closest contact may move throughout the evening as the planes approach each other.

In order to establish the point of closest contact, and thus the place where the conduit can currently be opened, you must follow these steps:
\begin{enumerate}
  \item Maneuver this machine to be directly in front of ``A Possible Location of Closest Contact'' Sign.
  \item {\bf Only once step 1 has been completed}, lift the first page of the ``possible location'' sign up to read the chart on the second page. Locate the number associated with the current time. This is the active number.
  \item Check whether the number displayed on the machine is a factor of the active number on the sign. If the number on the machine is a factor of the number on the sign, then this location is currently the point of closest contact and the conduit can be opened from this location - if you know how. (For example, if the machine display reads ``5'' and the active location number was ``20'' then that location would be the point of closest contact and the conduit {\bf could} be opened here. But if the active number was ``23'', then the location would not be the point of closest contact and the conduit {\bf could not} be opened here.) \emph{You can't check numbers that aren't currently active.} 
\end{enumerate}

\end{document}
%%Machine has 1 number and instructions
%%Each conduit location has 2 pages. covered page has list of numbers that are active at different times
%%Check for prime(?) factors