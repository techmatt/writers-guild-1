%%%%%
%%
%% GameTeX will input the file run\gamerun-LIST.tex.  So if \gamerun
%% is set to 1, run1-LIST.tex will be used, if 2, run2-LIST.tex, etc.
%% The file is not required; if it does not exist, no error will
%% occur.  run1-LIST.tex is likely unnecessary.
%%
%% You can use this file for run-dependent modifications (usually
%% player information in Char datatypes).
%%
%% \updatemacro{<datatype macro>}{<new updates>}
%% \updateplayer{<Char macro>}{<new player>}
%% \updateplayeremail{<Char macro>}{<new player>}{<new email>}
%%
%% \updatemacro will extend the contents of a datatype macro.  Use \rs
%% to change field values:
%%
%%   \updatemacro{\cTest}{
%%     \rs\MYplayer	{New Player}
%%     \rs\MYemail	{test@other.org}
%%     \rs\MYaddress	{Dorm, rm 201}
%%     \rs\MYphone	{x1-1111}
%%     \rs\MYsex	{\female}
%%     }
%%
%% \updateplayer is a shortcut for changing just the player name for a
%% character:
%%
%%   \updateplayer{\cTest}{New Player}
%%
%% \updateplayeremail is a shortcut for changing just the player name
%% and player email address for a character:
%%
%%   \updateplayeremail{\cTest}{New Player}{test@other.org}
%%
%%%%%


\updatemacro{\cActive}{
  \rs\MYplayer          {David Gupta}
}

\updatemacro{\cBride}{
  \rs\MYplayer          {Sharon Beltracchi}
}

\updatemacro{\cGroomA}{
  \rs\MYplayer          {Jeremy Cole}
}

\updatemacro{\cGroomB}{
  \rs\MYplayer          {David Green}
}

\updatemacro{\cKid}{
  \rs\MYplayer          {Daniel Kane}
}

\updatemacro{\cPassive}{
  \rs\MYplayer          {J}
}

\updatemacro{\cPastor}{
  \rs\MYplayer          {Peter Ciccolo}
}

\updatemacro{\cProgrammer}{
  \rs\MYplayer          {Darius Johnston}
}

\updatemacro{\cRival}{
  \rs\MYplayer          {Nick Zehender}
}

\updatemacro{\cScientist}{
  \rs\MYplayer          {Sarah Terman}
}
