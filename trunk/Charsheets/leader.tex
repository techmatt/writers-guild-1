\documentclass[char]{elementals}
\begin{document}
\name{\cLeader{}}

You are President \cLeader{\intro}, the elected leader of the meta-government formed to replace the New United Nations when it was officially disbanded at the end of the 22nd century. You have your hordes of detractors and a steady base of supporters. You also find that you have to keep reminding people who blame you for every ill they experience that seeing as you possess no veto or legislative powers, you have considerably less influence over matters than they think. Most of your day-to-day activities involve keeping faction A from engaging in physical altercations with faction B because they refuse to censure faction C who doesn't agree with A's viewpoints on issue D. That said, you are still the elected representative of a good fraction of the human population, and when the need arose for a single individual to represent humans at the gathering today, you're it.

Your general approach to governance is mediation. When two forces oppose each other, you try your best to defuse the situation by whatever means possible. Although it's tricky, thus far you've managed to apply this philosophy even when the elementals are concerned. Whenever elementals wreak havoc -- and they do, all the time -- there is always a call for all out war with whatever faction of elementals caused the problem (or all elementals at once, because why not?). Your military advisors give you almost daily reminders that this would be a Very Bad Idea, and up until now you have done a good job of pacifying people. That said, you have also never really managed to extract any significant concessions or even apologies from the elementals involved: destruction of property or even murder seem like practically alien concepts to them. Still this has gotten quite a bit better over the past 50 years or so. Improvements in agriculture, for example, have made modern farming quite a bit more resistant to a pack of fire elementals deciding that they want to bathe in the burning remains of a few thousand acres of farmland.

While things are a bit better than they were 100 years ago, they are still not good. The main reason you entered politics in the first place was because you feel very strongly that the reason elementals act so boldly towards humans is that they know there are no real consequences for doing so. Elementals simply don't fear death or war on any scale the way that humans do. They do, however, fear each other. You are confident that if you could convince one of the elemental factions to support you, then humanity would be seen as a force to be left alone.

Despite your efforts, most of the elemental factions simply ignore humans, and when pestered resort to enslaving your diplomats for amusement or worse. But your persistence has eventually paid off: \cKing{\intro}, the king of the air elementals, has agreed to establish persistent diplomatic relations with humans and is the reason you were invited to the meeting today. Unfortunately, he has thus far avoided agreeing to anything close to an official treaty -- he claims that while he finds humans interesting, they really have not proven that they have anything to offer air elementals in return for his protection. You are determined to prove him wrong, and this meeting is the perfect opportunity to do so. That said, many other elemental factions will be present and you will forge a treaty with however many factions you can.

Years of political engagements have taught you the wisdom of doing your research \emph{before} you come to a meeting like the one you're in charge of today. From what your spies have managed to discern, three different elemental factions will be present: fire, air, and earth. Water seems to be concerned about the situation but is keeping an eye from a distance.

%Idea: what's below could be literally told in the form of a ``report on the elemental factions'' whitesheet?

Fire has always been the faction that frightens you the most. The formidable \cQueen{\intro}, the queen of the fire elementals and one of the original four elementals who came to Earth 100 years ago, will be in attendance with a sizable contingent of servants. You have only been ``invited'' to an audience with her once before, and you wouldn't care to repeat the experience. No one really seems to have a great idea what \cQueen{} plans to accomplish today, but you figure you had better try and find out. You are also a bit terrified by the fact that in over a century of observing them, humans have never before witnessed two of the elemental leaders meeting in one place.

As best you can tell, the earth king will not be present; there is a rumor he is ``sick'', which is a bit puzzling to you and you have made a mental note to ask some of your lead scientists more about how elemental physiology works. Still, several earth elementals will be present today, although with the king absent the hierarchy is unclear. You believe the leading earth elemental is called \cLoyal{\intro}, but humans have very few diplomatic ties to earth elementals and you don't know much beyond that.

Given the significant dangers posed by the elemental factions, you are acutely aware the people attending with you need to be united, and as the leader the task of making sure that happens falls on you. There are seven other humans coming with you, most of whom you already know. First there's your bodyguard \cRomeo{\intro}, an Azi soldier who you generally get along with. As is standard procedure for Azi bodyguards, you are also \cRomeo{}'s handler. Next there's \cDema{\intro} who is the only other member of your executive branch present. \cDema{\They} is a talented speaker who is good at playing up the emotions on crowds; although your relationship is complicated, in general \cDema{\they} serves as a good foil to your more conciliatory negotiation style. Also accompanying you are \cAvatar{\intro}, the head diplomat in charge of negotiating a peace treaty with the elementals, and \cAvatar{\their} Azi assistant \cDiplomat{\intro}, also a talented negotiator.

Accompanying the politicians is a sizable contingent of scientists. Officially all three scientists report to you, although in practice you've found scientists tend to pay little attention to the chain of command. The most infamous of the scientists is \cGD{\intro}, the great granddaughter of Konrad Strauss, the scientist who originally got us into this mess 100 years ago. She's apparently made quite a name for herself in the scientific community, and specializes in researching the Conduit and the elemental plane. Also present is \cMS{\intro}, a more eccentric scientist who focuses on the elemental's mental powers. Finally, there is \cScientist{\intro}, an Azi scientist who your report indicates is (as expected of an Azi) a stable and capable individual who studies the chemical processes that underlie the elementals. Your reports indicate that there are fairly tense social interactions between the three scientists, and that \cMS{} has been accused of safety procedure violations in the past by \cScientist{}. Still, all three are present because even after 100 years no one fully understands what exactly makes elementals tick, and these are the three people best situated to learn more. In particular, the opportunity to study one of the elemental leaders in close proximity has almost never presented itself before. You're just going to have to keep an eye on these three to make sure they don't get incinerated or cause an inter-species war. You also suspect the scientists have their own research agendas and you are going to have to find out what exactly they are.

In a matter of unfortunate necessity, your security team spies on humans as well as elementals. Given that elementals have the ability to mentally control humans, interactions between elementals and people who are ``politically or scientifically sensitive'' are supposed to be closely monitored. However, your spies have reported that there is significant evidence that one or more of the humans attending with you is having unsupervised interactions with elementals. This could indicate any number of things, but the more disturbing implications are that one or more of the members of your entourage are either willingly working with the elementals or are actively in thrall to one of the elemental leaders. You do not like to think about what this means.

%does leader have some ``resist enslavement'' tech from MS?

%TODO: airplane incident

\begin{itemz}[Goals]
  \item Maintain peace between humans and the elementals at all costs.
  \item Keep all the humans attending the gathering alive and happy.
  \item Work with \cAvatar{} and \cDiplomat{} to negotiate a peace treaty with one or more of the elemental factions.
  \item You do not know what the various elementals are planning to do here; find out. Whatever it may be, prevent humanity from being marginalized or worse by the elementals.
  \item Figure out what \cGD{}, \cMS{}, and \cScientist{} are up to and make sure it isn't too dangerous.
  \item Investigate humans that may be actively enslaved by elementals or working with them in secret.
\end{itemz}

\begin{contacts}
  \contact{\cRomeo{}} An Azi soldier serving as your bodyguard.
  \contact{\cDema{}} Your 2nd-in-command. TODO - more details.
	\contact{\cAvatar{}} The head Cit diplomat in charge of helping you forge a peace treaty with the elementals. Talented, but often loses focus and has a rather absent-minded personality.
	\contact{\cDiplomat{}} An Azi negotiator who reports to \cAvatar{}.
	\contact{\cGD{}} The great granddaughter of Konrad Strauss. Specializes in the elemental plane and the Conduit.
	\contact{\cMS{}} An eccentric Cit scientist who specializes in understanding how elementals can enslave humans and each other. Has a history of procedural violations.
	\contact{\cScientist{}} A talented Azi scientist who researches elemental biochemistry.
	\contact{\cQueen{}} The queen of the fire elementals.
	\contact{\cKing{}} The king of the air elementals.
	\contact{\cNaturalist{}} \cKing{}'s trusted advisor, a high-ranking air elemental.
	\contact{\cLoyal{}} A high-ranking earth elemental.
\end{contacts}

\end{document}
