\documentclass[char]{elementals}
\begin{document}
\name{\cNaturalist{}}

For as long as you can remember, you have been curious.  Curious about the wind patterns on which you float, curious about the lives of other elementals, curious about this strange world in which you were born, and curious about the elemental plane in which you should have existed.  For nearly a century, you have travelled the world and seen its wonders, and yet there is still more to see and know.  Of course, being an air elemental does give you some advantages in being able to move around the world; you can float where the wind takes you, and the wind will invariable take you where you want to go.  You had seen so much of the human world that you were the natural first choice of your King, \cairKing{\intro} when he needed someone to assist him learn more about humans.  For some time, you had watched them from afar, unsure of whether your interference would harm them, but when your King asked you to speak to them, you delighted in the opportunity.

%%paragraph about previous research but things you still don't know, will have a better idea when I am thinking about the research notebook.  

Of the humans, the scientists always seemed to be the most willing to talk to you.  You learned that some of them are even studying elementals, and were as interested in learning about you as you were in learning about them.  That is how you met \cgrandDaughter{\intro}.

%%paragraph about the conduit, how you know granddaughter, not sure if you approve of what air king wants to do with it... to be continued

Of course, your studies have not always gone perfectly.  Recently, you were en route to report back to your King, when you saw two large metal birds in the air.  You decided to have a closer look.  They seemed not to be birds, but large, winged metal tubes, which had windows through which you could see human faces.  You paused for a moment, marveling at how these tiny, mundane humans had been able to nevertheless build a flying machine.  You watched the machines soar through the clear sky for a moment, and began to wonder if these human contraptions were carried by the winds in the same way Air elementals were.  You began to manipulate the air currents, and watched the planes smoothly move up and down, side to side, for a time not noticing the look of panic that was overtaking the human faces inside.  It was only when you managed to make one machine roll over to see if it could fly upside-down that you realized you were taking things too far.  However, startled by your own realization that you might be hurting the little humans inside, you breifly stopped paying attention to what you were doing with the wind, and the two flying machines collided with a mighty explosion before hurtling into the ocean below.  You fled the scene in a panic and confessed all to \cairKing{}.  \airKing {They} at first did not understand why you were so upset, but eventually \airKing{they} came to see that if the human leader found out you had crashed the airplanes, the air elemetals would be unlikely to be able to pursue a treaty and make peace with the humans.  \cairKing{} suggested you blame a low-ranking fire elemental.  There were enough flames in the crash, and Fire is responisible for a great deal of destruction of human buildings, it seemed plausible.  You knew that \cJuliet{intro} was travelling a similar route that day, and could have easily come across the flying machines.

For now, you are content to keep point the blame on \cJuliet{them}.  If the humans found out you were to blame, it would not only make what the other air elementals were trying to do more difficult, but might also bar you from further research with the humans.  Still, you cannot get the image of the terrified human faces in the airplane windows out of your head.  The guilt weighs heavily on your conscience, but you remind yourself that science is not without risk.

%%more on feelings about crash, leading into info on poison Earth King plot

\end{document}
