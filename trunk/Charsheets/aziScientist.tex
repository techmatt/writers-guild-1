\documentclass[char]{guildcamp1}
\begin{document}
\name{\cScientist{}}

You are \cScientist{\intro}, an Azi who specializes in scientific research. Even more than most Azi, you really enjoy what you do. Nothing makes you happier than to take a new problem or unexplained phenomenon, formulate hypothesis about how to solve the problem or explain the behavior, run experiments to validate or reject the hypothesis, and eventually form explanations or solutions to the problem. You are also a model team player -- you collaborate well with other scientists, you are a great reviewer, and your explanations are clear and easily understandable by others. Others seem to frequently commend you, but you tend to just nod appreciatively; it should really come as no surprise that you have all these positive qualities. They are, after all, a fundamental part of the tape you were originally programmed with. You like to think that you would have all these properties even if you were a CIT, but you also aren't bothered by being an Azi. You appreciate the simplicity of both knowing exactly what you want to do and being very good at it, and even given the option you wouldn't want to change that. Your CIT handler is Galina Schwartz, a scientist who is currently working at a lab in Toronto.

Your research is all over the map, but your best area is in chemical analysis and synthesis. You are renowned for your ability to discern the functions of complex molecules or proteins and you have your name attached to more than one synthesis pipeline for compounds previously believed to be nearly impossible to fabricate. Your one stumbling block is elementals -- try as you might, there are many chemical processes that go on within the elemental ``species'' that even after 100 years continue to defy explanation. That said, research is ongoing and you are still making progress, if slowly. Naturally, you expect to gather large amounts of valuable data at the Conduit site today. Analyzing the data will take time but you are definitely looking forward to it. You might even be able to find some willing elementals to perform some basic tests on. While humans have had plenty of chances to interact with elementals directly, due to their physical and mental capabilities, it's quite dangerous and not allowed as often as you'd like.

You are a strong believer in the division of labor, and science is your domain, not politics or bureaucracy. Still you realize that occasionally the matters you investigate have political ramifications, and towards that end you have a close friendship with \cDiplomat{\intro}, a talented Azi diplomat. Fortunately, \cDiplomat{} was also selected for the human delegation to the Conduit, and you can't really think of anyone better for the task. When you make discoveries that you think might have political repercussions, you always discuss them with \cDiplomat{} and trust that \cDiplomat{\they} will take the most prudent action and inform the relevant figures.

Part of your scientific training involves firm grounding in how to ethically conduct experiments. This has become quite a large concern given that many in the political community are pushing for less strict restrictions on experiments in light of the elemental threat. However, after a large public backlash from a number of extremely high-profile experiments that resulted in the death of several Azi became publicized, there has been reinforced support for even stricter guidelines to make sure these incidents don't repeat themselves. To you, as a core part of your programming, conducting experiments ethically is a given and you always err on the side of caution rather than trying to find clever loopholes. This ethical compulsion tends to go beyond even what most scientists would consider sufficient -- you would rather have your work destroyed than see it used in a manner you felt had the potential to be misused. You go to great lengths to make sure that what you work on will be used appropriately and often discuss such issues with \cDiplomat{} to gain a broader perspective.

You get along quite well with \cGD{\intro}, the great granddaughter of Konrad Strauss, who shares many of your interests and concerns. On the other hand, \cMS{} is nothing but trouble. At best, \cMS{\they} is often lax with reporting \cMS{\their} experiments for external approval. At worst, \cMS{\they} may have actively conducted secret experiments because \cMS{\they} knew that they would be blatantly rejected by the scientific community. \cMS{\They} has on at least one occasion attempted to get you to assist with such covert research; you suspect it was because \cMS{\they} believes Azi scientists are interested only in research and incapable of percieving what \cMS{\they} is actually doing. Naturally, you reported \cMS{\their} violations and was never asked to collaborate again. But you are not foolish and are quite aware \cMS{\they} is likely to continue such experiments. If \cMS{\they} does, you will find out and work with \cDiplomat{} to have \cMS{\them} taken care of.

When you arrived at the Conduit only a few minutes ago, you wasted absolutely no time investigating before a large influx of humans and elementals contaminated the site. Upon rapid examination of some of the smashed apparatus in Konrad Strauss's lab, one tiny speck of white dust seemed out of place. It looks like white sand, although there are only a few grains. However, after running it under your mass spectrometer, you were stunned by the resulting spectrograph. Normally, a mass spec triggers only a few peaks at the presence of a small number of different mass-to-charge ratios, but this substance seemed to be radiating at virtually every charge and mass combination in huge amounts wildly inconsistent with the mass of the input sample. After convincing yourself that your mass spectrometer was not broken, you became a bit overwhelmed at the possible ramifications of such a substance. It is a much, much more intense form of some of the behaviors scientists have noticed regarding aspects of elemental chemistry, and if you can understand it, you are confident it has the potential to explain questions you have been pondering for the majority of your career. In general you are still pretty buzzed from your discovery, but your more cautious side is also very concerned that you have so little of the substance. You are not going to assume that such a substance is connected to the Conduit, but it certainly seems very likely; either way, you are determined to find out what the substance is, where it came from, what it does, and how to synthesize more for further study.

\end{document}
