\documentclass[char]{elementals}
\begin{document}
\name{\cQueen{}}

You love this new world.  Even after a hundred years, it continues to amaze you by the opportunities it presents --- such different terrains, ranging from the desolate ice plains and endless deserts to golden prairies and lush green forests.  And, oh, so many volcanoes to occupy and turn into palaces!  Without any interference from those pesky Quintessence creatures, you are now free to realize your full potential --- as soon as you conquer this entire planet with the Fire elemental army you intend to create.  Every once in a while, the elemental plane and earth are aligned closely enough to open a Conduit between them.  Even though you have proclaimed from your fiery palace that you would like to open it in order to bring over elementals of all factions, you have no intention of actually doing so.  Instead, you're planning to pull over as many Fire elementals as possible so you can wipe out the other factions!  After all, you are \cQueen{}, the daughter of Blaze, the daughter of Gehenna, the daughter of the mighty Vul.  The whole earth should bow to your glory!

The pre-existing inhabitants, those creatures calling themselves ``humans'' who brought you over from the elemental plane in the first place and then regretted it, still present the occasional annoyance but thankfully their numbers are much diminished now.  Much of the credit for this housekeeping goes to your subordinate \cPyro{\intro}, who is convinced that the Fire elementals should conquer the entire earth, and who is zealous in removing humans who get in the way.  For example, when you first established yourself in your primary volcanic home, there were still humans living in settlements of ugly squarish structures nearby.  They quite spoiled the view.  \cPyro{} was the one who suggested burning them out using lava from the volcano.  Once you approved the plan, \cPyro{\they} carried it out beautifully, pouring tons of lava and fire upon the settlement.  The humans haven't returned since to bother you.

However, you are concerned that they are adapting to the elemental presence on earth and that they will ally with one of the other factions, Water, Air, or Earth, to strengthen it.  Right now, all four elemental factions are vying for power.  Earth presents the least immediate threat, because its King is seriously ill and his two deputies disagree with each other.  Actually, some rumors are circulating that he is dying, which would weaken that faction even more.  However, you've noticed that your own powers have gradually been decreasing --- you are not as powerful as you were a hundred years ago.  This may be because you are tired from creating Fire elementals to serve you, but you suspect that there is more to it.  You worry that the Earth King's illness is related to your own waning powers.  That, of course, is a longer term problem --- but great leaders are supposed to think ahead, aren't they?

The second greatest immediate threat is Water, your old enemy, which has ensconced itself quite firmly in the many oceans (as soon as you take over the earth, you're going to boil them all away).  You haven't seen the Water Queen for some time now, so you're certain she is plotting against you.  Actually, you think this latest conflict is all \cPyro{}'s fault: recently, a lowly Water minion dared to contradict \cPyro{\them}, and \cPyro{} killed it for being uppity.  The Water Queen immediately retaliated by inundating one of your nicest volcanic islands with a massive tsunami.  What a mess!  A properly repentant \cPyro{} oversaw repairs, of course, but you're still angry at \cPyro{\them} for \cPyro{\their} impetuousness.  Faction pride is all well and good, but not when it results in inconveniences to \emph{you}.  Anyway, you are wary of Water and what it may do.

As you see it, Air is the greatest threat to your ascendance.  Being silly and fickle, Air alternately considers humans to be amusing pets or potential allies.  Right now their king, \cKing{\intro}, is toying with the idea of an alliance with them and frequently meets with their leader, \cLeader{\intro}.  The human seems pretty desperate to sign a peace treaty with as many elemental factions as possible (because somehow \cLeader{\they} thinks a piece of paper would keep you from incinerating inconvenient humans?).  You have received information about humans from another source as well: the humans' second-in-command, Secretary of State \cDema{\intro}, who wants to become supreme leader of the humans.  You see in \cDema{\them} a kindred spirit and have grudging respect for \cDema{\their} ambition.  Of course, you are perfectly aware that you cannot trust everything \cDema{\they} says, but according to \cDema{\them}, the humans have been developing potent technology that is harmful to elementals.  \cDema{\They} volunteered this warning in hopes of winning your support for \cDema{\their} power bid, which does lend credence to it.  You remain suspicious of \cDema{\their} claims, but if \cDema{\they} is correct, this knowledge will give whichever faction allies with the humans an edge over the other factions.  It may even tip the balance.  You're not convinced that you want to \emph{ally} with the lowly humans, but you do need their knowledge to open the Conduit (at least, you don't know how to and they do --- or so \cDema{} claims).  As such, you definitely want to prevent an alliance between Air and the humans.  Before you can decide, you need more concrete information.

 To insinuate \cPyro{\themself} back into your good graces, \cPyro{} recommended \cJuliet{\intro}, a relatively weak Fire elemental of whom \cPyro{} is quite fond, as a spy.  You named her emissary to the humans and gave her extra Fire powers on a provisional basis, and so far \cJuliet{\they} has proven useful.  If \cJuliet{\they} continues to be successful, you plan to make \cJuliet{\their} promotion permanent and put \cJuliet{\their} in charge of training more intelligence operatives.  You are mildly surprised that \cJuliet{\they} seems to get along with the humans reasonably well, but that just makes \cJuliet{\them} a more effective spy, so that's okay.  After all, knowledge is power, and power is what you want!




\begin{itemz}[Goals]
  \item Open the Conduit and bring over only Fire elementals. You desire to acquire both lesser servants through a \emph{momentary opening} and greater servants through a \emph{terminal opening}.
  \item Find out what anti-elemental military technology the humans are developing.
  \item Find out what the Water elementals are up to.
  \item Prevent a human-Air alliance.
  \item Meet with the Air King, \cKing{\full}, to see if his powers have been waning too.  In secret, of course, because you wouldn't want your minions to find out you're getting weaker!
\end{itemz}

\begin{contacts}
  \contact{\cKing{}} King of the Air elementals who may sign a peace treaty with humans.  Your rival for power among the elementals.
  \contact{\cPyro{}} A subordinate who is militant and at times over-zealous about imposing Fire authority.
  \contact{\cJuliet{}} A Fire elemental who is your spy.  \cJuliet{\Their} cover is acting as intermediary between you and \cKing{}, as well as your official emissary to the humans.
  \contact{\cDema{}} Secretary of State of the humans who wants to become supreme leader with your support.
\end{contacts}


\end{document}
