%%%%%
%%
%% This file sets up the Green datatype and creates Green macros.
%% These are for greensheets.
%%
%%%%%

\DECLARESUBTYPE{Green}{Element}
\PRESETS{Green}{
  \FD\MYfile	{} %% filename (in \greens dir)
  \FS\MYtext	{\ifx\MYfile\empty\else%
		  \getextractenvs{document}{\greens/\MYfile}%
		\fi}
  }


%%%%%%%%%%%%%%%%%%%%%%%%%%%%%%%%%%%%%%%%%%%%%%%%%%%%%%%%%%%%%%%%%%

\NEW{Green}{\gTest}{
  \s\MYname	{Test Greensheet}
  \s\MYfile	{README.tex}
  }


%%%%%%%%%%%%%%%%%%%%%%%%%%%%%%%%%%%%%%%%%%%%%%%%%%%%%%%%%%%%%%%%%%

\NEW{Green}{\gConduitCover}{
  \s\MYname	{How to Use This machine}
  \s\MYfile	{ConduitCoverSheet.tex}
  %%The \nConduitLocations{} RN needs to end up associated with this GS some how
  }

\NEW{Green}{\gConduit}{
  \s\MYname	{How to Open the Conduit}
  \s\MYfile	{OpenTheConduit.tex}
  }

\NEW{Green}{\gConduitModifier}{
  \s\MYname	{Modifications to Conduit Opening}
  \s\MYfile	{ModifyTheConduit.tex}
  }


%%%%%%%%%%%%%%%%%%%%%%%%%%%%%%%%%%%%%%%%%%%%%%%%%%%%%%%%%%%%%%%%%%%%%

