 %%%%
%%
%% This file sets up the Sign and Label datatypes and creates Sign and
%% Label macros.
%%
%% Signs generally represent interesting parts of game area, usually
%% as things posted on walls.  Labels represent other things, often on
%% or inside envelopes, that are part of complex mechanics.
%%
%% The default value for \MYloc will inherit location from the Place
%% or Sign most immediately up the ownership tree.  Override this by
%% setting \MYloc to anything (even blank).
%%
%% Sign is for full-sized signs that would cover most of a large
%% manila envelope; SignMedium is for signs sized to half-sized manila
%% envelopes; SignSmall is for signs sized for small manila envelopes
%% (the same size as item cards).  Label, LabelMedium, and LabelSmall
%% are analagous, but they don't have a \takedownby note at the
%% bottom.  You can always use a sign or label without an envelope or
%% with a differently-sized envelope.  Choose which based on
%% visibility and content.
%%
%% SignTiny is for signs you want to be hard to find; it is small and
%% does not have a \takedownby note.  SignDot is for a very small
%% "dot" which only has a title.
%%
%% SignStrip produces a strip of paper (without a \takedownby note)
%% with labels on the outside that show on both sides if you fold it
%% in half.  These are a convenient alternative to sub-envelopes. They
%% can also be used for "s-packets" taped to walls (see
%% Extras/README-s-packets).
%%
%% LabelCover produces a label similar to the cover to a research
%% notebook.  LabelPage, likewise, produces a page.
%%
%% EOG is for full-sized end-of-game signs.
%%
%%%%%

\DECLARESUBTYPE{Sign}{Element}
\PRESETS{Sign}{
  \FD\MYloc	{\mylocation} %% real-space location
  \FD\MYtext	{} %% text of sign
  }
\POSTSETS{Sign}{
  \edef\mylocation{\MYloc}
  \protected@edef\@ownerstring{%
    \MYname%
    \ifx\mylocation\empty\else\ (\mylocation)\fi%
    }
  }
\def\mylocation{}

\def\loc#1{\rs\MYloc{#1}}

\DECLARESUBTYPE{SignMedium}{Sign}
\DECLARESUBTYPE{SignSmall}{Sign}
\DECLARESUBTYPE{SignTiny}{Sign}
\DECLARESUBTYPE{SignDot}{Sign}
\PRESETS{SignDot}{\s\MYtext{}}

\DECLARESUBTYPE{Label}{Sign}
\PRESETS{Label}{\s\MYloc{}}
\DECLARESUBTYPE{LabelMedium}{Label}
\DECLARESUBTYPE{LabelSmall}{Label}

\DECLARESUBTYPE{SignStrip}{Sign}
\DECLARESUBTYPE{LabelCover}{Label}
\DECLARESUBTYPE{LabelPage}{Label}

\DECLARESUBTYPE{EOG}{Sign}
\PRESETS{EOG}{%
  \s\MYname	{End Of Game}
  \s\MYtext	{{\bf\Huge You may not pass through here.}}
  }


%%%%%
%% \signbig[<location>]{<name>}{<text>}
%% \eog[<location>]
%%
%% \signmdeium[<location>]{<name>}{<text>}
%% \signsmall[<location>]{<name>}{<text>}
%% \signtiny[<location>]{<name>}{<text>}
%% \signdot[<location>]{<name>}
%%
%% \labelbig{<name>}{<text>}
%% \labelmedium{<name>}{<text>}
%% \labelsmall{<name>}{<text>}
%%
%% \signstrip[<location>]{<name>}{<text>}
%% \labelcover{<name>}{<text>}
%% \labelpage{<name>}{<text>}
\newinstance{Sign}{\signbig[3][\mylocation]}{
  \s\MYloc{#1}\s\MYname{#2}\s\MYtext{#3}}
\newinstance{EOG}{\eog[1][\mylocation]}{\s\MYloc{#1}}

\newinstance{SignMedium}{\signmedium[3][\mylocation]}{
  \s\MYloc{#1}\s\MYname{#2}\s\MYtext{#3}}
\newinstance{SignSmall}{\signsmall[3][\mylocation]}{
  \s\MYloc{#1}\s\MYname{#2}\s\MYtext{#3}}
\newinstance{SignTiny}{\signtiny[3][\mylocation]}{
  \s\MYloc{#1}\s\MYname{#2}\s\MYtext{#3}}
\newinstance{SignDot}{\signdot[2][\mylocation]}{
  \s\MYloc{#1}\s\MYname{#2}}

\newinstance{Label}{\labelbig[2]}{
  \s\MYname{#1}\s\MYtext{#2}}
\newinstance{LabelMedium}{\labelmedium[2]}{
  \s\MYname{#1}\s\MYtext{#2}}
\newinstance{LabelSmall}{\labelsmall[2]}{
  \s\MYname{#1}\s\MYtext{#2}}

\newinstance{SignStrip}{\signstrip[3][\mylocation]}{
  \s\MYloc{#1}\s\MYname{#2}\s\MYtext{#3}}
\newinstance{LabelCover}{\labelcover[2]}{
  \s\MYname{#1}\s\MYtext{#2}}
\newinstance{LabelPage}{\labelpage[2]}{
  \s\MYname{#1}\s\MYtext{#2}}


%%%%%
%% \sEOG{}
%% use \sEOg[\loc{<location>}]{} for EOG sign at a specific place
\NEW{EOG}{\sEOG}{
  }


%%%%%%%%%%%%%%%%%%%%%%%%%%%%%%%%%%%%%%%%%%%%%%%%%%%%%%%%%%%%%%%%%%

%\NEW{Sign}{\sTest}{
%  \s\MYname	{A Room}
%  \s\MYloc	{10-250}
%  \s\MYtext	{A lecture hall with large, sliding blackboards.}
%  }


%%%%%%%%%%%%%%%%%%%%%%%%%%%%%%%%%%%%%%%%%%%%%%%%%%%%%%%%%%%%%%%%%%

%%Cart Signs
\NEW{Sign}{\sConduitLocatingNumber}{
  \s\MYname	{Machine Display}
  \s\MYloc	{}
  \s\MYtext	{{\Huge 487}}
}

%%GD Lab Signs 
  \NEW{Sign}{\sGDLab}{
  \s\MYname	{Katherine Strauss' Lab Space}
  \s\MYloc	{}%%TBD
  \s\MYtext	{This is a laboratory filled with a wide array of complex and only semi-portable machinery and equipment. It has evidently been set up in considerable haste. You cannot interact with this sign unless you know otherwise. 
  } 
}

\NEW{Sign}{\sRaidGD}{
  %my idea to make this as simple as possible would be that this is a double sheet that you can only look under after you raid the lab.
  \s\MYname	{Katherine Strauss' Lab Space}
  \s\MYloc	{}%%TBD
  \s\MYtext	{You scour through the lab's computers and experimental setups. You are fairly confident that \cGD{} is building a device that will try to send {\bf all} the elementals through the Conduit. The core component of this device is an electromagnet, although it can apparently only function when hooked up to an open connection to the elemental planes. It also requires considerable amounts of power to function.
  
  \emph{If \cGD{} is dead, you should contact a GM now.}
  } 
}

\NEW{Sign}{\sGDLabChair}{
  \s\MYname	{The Chair of Science!}
  \s\MYloc	{}%%TBD
  \s\MYtext	{A metal chair. Numerous wires run from it to near-by monitors. There are also wrist and ankle cuffs for securing the test subject. 
  
  The chair is crucial for certain experiments. It is bolted to the floor and cannot be moved or destroyed.
  
  \emph{If a player is instructed to sit in the chair, they should role play accordingly.}
  } 
}
  
\NEW{Sign}{\sGDLabBench}{
  \s\MYname	{Katherine Strauss's Lab Bench}
  \s\MYloc	{-002}
  \s\MYtext	{This bench is littered with items relevant to doing scientific research.
  } 
  \s\MYitems{\iPhaseTransducer{} \iHeisenbergCompensator{} \iSiliconeMitt{}}
}

\NEW{Sign}{\sHighResMass}{
  \s\MYname	{A High Resolution Mass Spectrometer}
  \s\MYloc	{-002}
  \s\MYtext	{This machine is great for doing scientific research that requires high resolution.} 
  \s\MYitems{}
}
 
\NEW{Sign}{\sScale}{
  \s\MYname	{A Scale}
  \s\MYloc	{-002}
  \s\MYtext	{Used for weighing reagents.} 
  \s\MYitems{}
}
  
%%Azi Sci Lab
\NEW{Sign}{\sSciLab}{
  \s\MYname	{Percy Bentz's Lab Space}
  \s\MYloc	{}%%TBD
  \s\MYtext	{Beakers, test tubes, and chemical samples clutter the area. You cannot interact with this sign unless you know otherwise. 
  } 
}
 
\NEW{Sign}{\sRaidSci}{
  %my idea to make this as simple as possible would be that this is a double sheet that you can only look under after you raid the lab.
  \s\MYname	{Percy Bentz's Lab Space}
  \s\MYloc	{}%%TBD
  \s\MYtext	{You scour the lab's computers and experimental setups. \cScientist{} keeps very meticulous notes and it is easy to follow \cScientist{\their} line of work. After discovering a strange white powder, \cScientist{} began investigating its properties. It appears to be tightly connected to the elemental biological process, and is possibly related to the steady loss of elemental energy levels. There does not appear to be much else that \cScientist{} has been working on lately, besides some recently published papers.
  
  \emph{If \cScientist{} is dead, you should contact a GM now.}
  
  } 
}

\NEW{Sign}{\sTransmogrifier}{
  \s\MYname	{A Matter Transmuter}
  \s\MYloc	{-002}
  \s\MYtext	{An overly-complicated device that funnels a large set of lasers into a central vacuum chamber. 
  
  {\bf You cannot interact with this device unless you know otherwise.}} 
  \s\MYitems{\multi{10}{\iWhite{}}}%%White lives in the ``A'' Packet.
}

\NEW{Sign}{\sAziSciLabBench}{
  \s\MYname	{Percy Bentz's Lab Bench}
  \s\MYloc	{-002}
  \s\MYtext	{This bench is littered with items relevant to doing scientific research.
  } 
  \s\MYitems{\iHeisenbergCompensator{} \iAirplanePiece{} \iSiliconeMitt{}}
}

\NEW{Sign}{\sPeriodicTable}{
  \s\MYname	{The Periodic Table of Elements}
  \s\MYloc	{-002}
  \s\MYtext	{There is a periodic table of the elements here. Hydrogen... Helium... Lithium... Beryllium....} 
  \s\MYitems{}
}

\NEW{Sign}{\sDistilling}{
  \s\MYname	{A Large Distilling Apparatus}
  \s\MYloc	{-002}
  \s\MYtext	{Crucial for ensuring that water used for experimenting is pure.} 
  \s\MYitems{}
}

%%Mad Scientist Lab Signs
\NEW{Sign}{\sMSLab}{
  \s\MYname	{Sofia Alexandrov's Lab Space}
  \s\MYloc	{}%%TBD
  \s\MYtext	{Throughout this room are hastily set up computers and mechanical apparati in various stages of assembly. You cannot interact with this sign unless you know otherwise. 
  } 
  }
  
\NEW{Sign}{\sRaidMS}{
  %my idea to make this as simple as possible would be that this is a double sheet that you can only look under after you raid the lab.
  \s\MYname	{Sofia Alexandrov's Lab Space}
  \s\MYloc	{}%%TBD
  \s\MYtext	{You scour the lab's computers and experimental setups. There is quite a lot of highly encrypted or hastily erased data here, but you can still manage to recover some of the projects. \cMS{} is primarily researching the process by which elementals enslave humans or elementals, and is trying to replicate it. So far as you can tell, the focus is only on enslaving elementals and not humans. So far, \cMS{\they} has managed to create a prototype device in the form of a long rod that can control very weak elementals, but there is also considerable work into Azi tape. \cMS{} has apparently created a new type of tape that can train Azi to enslave much more powerful elementals, although in its current form such tape is very dangerous (and illegal) to use. There is some evidence suggesting that a prototype version of this tape has already been experimented with several years ago.
  
  \emph{If \cMS{} is dead, you should contact a GM now.}
  
  } 
  }

\NEW{Sign}{\sLockedDoor}{
  \s\MYname	{A Steel Reinforced, Locked Door}
  \s\MYloc	{-015}
  \s\MYtext	{This steel door is locked.
  
  There are two ways to pass this door:
\begin{enumerate}
 \item Possess the appropriate key (item number: \iLabKey{\MYnumber}).
 \item Hit the door with a {\bf CR of 10} or greater. If you choose to do so, you must write ``Broken'' on this sign indicating that the door is now unlocked. Anyone may now freely pass into the room since the door has been broken down. {\bf Further,} elementals that participate in breaking the door down must write their elemental type on the back of sheet since this action leaves elemental residue on the door. Anyone possessing a \iTricorder{} with ``Detect Elemental Residue'' functionality may look under the sheet to learn the type(s) of elementals that participated in breaking the door down.
\end{enumerate}}
  \s\MYitems {}
}

\NEW{Sign}{\sCoffeeMachine}{
  \s\MYname	{A Coffee Machine}
  \s\MYloc	{-015}
  \s\MYtext	{This is an automatic coffee maker.
  
  {\bf If you would like to brew a cup of coffee, place 1 hand on this sign for 30 seconds. You may then take 1 coffee mug if there are any available.}}
  \s\MYitems  {\iCoffeeMug{} \iCoffeeMug{} \iCoffeeMug{}}
}

\NEW{Sign}{\sMSLabBench}{
  \s\MYname	{Sofia Alexandrov's Lab Bench}
  \s\MYloc	{-002}
  \s\MYtext	{This bench is littered with items relevant to doing scientific research.
  } 
  \s\MYitems{\iControlRod{}}
}

%%in all 3 labs
\NEW{Sign}{\sBatterySource}{%%3 sources in game, each has 3 batteries, for a total of 9 available in game.
  \s\MYname	{A Chemical Stock Room}
  \s\MYloc	{200-002}
  \s\MYtext	{This chemical stock room has sulfuric acid in it. You can build a battery here. To do so, take 2 minutes and consume 1 Copper (item number: \iCopper{\MYnumber{}}), and 1 \iZinc{} (tear them up), then take \iBattery{}. If this location has no batteries available, all of the sulfuric acid has been consumed and you must find another source.}
  \s\MYitems  {\multi{3}{\iBattery{}}}
}

%%Garden | Gathering Space
\NEW{Sign}{\sJanitorCloset}{
  \s\MYname	{A Caretaker's Cottage}
  \s\MYloc	{-013}
  \s\MYtext	{There is a Caretaker's Cottage tucked away at the edge of the gardens. An estate this big ought to have a caretaker to maintain the grounds, but it is clear that no one has lived here for years. Luckily, much of the caretaker equipment is still in working order.
  
  {\bf If you would like to take something from the cottage, place 1 hand on this sign for 30 seconds, then take the item of your choice.}}
  \s\MYitems  {\multi{3}{\iLye{}} \iWireTightener{} \iBucket{}}
}


\NEW{Sign}{\sAbilitySource}{
  \s\MYname	{$\Omega$ Packet}
  \s\MYloc	{-013}
  \s\MYtext	{{\bf You may not interact with this sign unless you know otherwise.}}
  \s\MYitems  {}
  \s\MYabils	{\multi{10}{\aWorkConduit{}}}
}

\NEW{Sign}{\sRoseBush}{
  \s\MYname	{A Rose Bush}
  \s\MYloc	{-013}
  \s\MYtext	{There is a ring of rose bushes here that have grown wild in the years without care. Even so, there are beautiful, large blooms all over the bushes. The air is gently perfumed by them.}
  \s\MYitems  {}
}

\NEW{Sign}{\sFountain}{
  \s\MYname	{A Stone Fountain}
  \s\MYloc	{-013}
  \s\MYtext	{A stone fountain sits here. It must have been lovely once, with a large bird bath, and a stone cherub. The cherub is cracked, and the fountain mechanism is clogged with leaves and is no longer functional.}
  \s\MYitems  {}
}

%%Cellar | Earth HQ
\NEW{Sign}{\sMalachite}{
  \s\MYname	{A Vein of Malachite}
  \s\MYloc	{-004}
  \s\MYtext	{There is a vein of malachite in the rock here. Malachite is a naturally occurring form of copper. 
  
  {\bf If you would like to take a piece of malachite, place 1 hand on this sign for 30 seconds.}}
  \s\MYitems {{\multi{10}{\iCopper{}}}}
}

\NEW{Sign}{\sGeodes}{
  \s\MYname	{A Pile of Rocks}
  \s\MYloc	{-004}
  \s\MYtext	{These rocks are worn smooth. They must have spent a lot of time in a river bed. Ancient river beds are a great place to find geodes.
  
  {\bf If you would like to take a rock, place 1 hand on this sign for 30 seconds.}}
  \s\MYitems {{\multi{3}{\iCrystalGeode{}}} {\multi{3}{\iRocks{}}}} %%50% chance of getting a geode
}

\NEW{Sign}{\sWineRack}{%%Flavor Text Sign!
  \s\MYname	{A Wine Rack}
  \s\MYloc	{-004}
  \s\MYtext	{There is a floor-to-ceiling wine rack. There are several dozen bottles of red wine here.
  
  {\bf If you would like to take \iWine{}, place 1 hand on this sign for 30 seconds, then take \iWine{}.}}
  \s\MYitems {{\multi{3}{\iWine{}}}}
}

\NEW{Sign}{\sBarrel}{%%Flavor Text Sign!
  \s\MYname	{A Huge Oak Barrel}
  \s\MYloc	{-004}
  \s\MYtext	{There is an old oak barrel sitting on a stand here. It is covered in dust, but there appears to still be something inside.
  
  {\bf If you would like to take \iMead{}, place 1 hand on this sign for 30 seconds, then take \iMead{}.}}
  \s\MYitems {\multi{3}{\iMead{}}}
}

\NEW{Sign}{\sRockSlide}{%%Flavor Text Sign!
  \s\MYname	{A Rock Slide}
  \s\MYloc	{-004}
  \s\MYtext	{A recent rock slide has taken out this part of the cellar. There are rocks scattered all over the floor, but none of them look very interesting.
  
  {\bf If you want a rock anyway, place 1 hand on this sign for 30 seconds, then take \iCrystalGeode{}.}}
  \s\MYitems {{\multi{1}{\iCrystalGeode{}}} {\multi{3}{\iRocks{}}}} %%25% chance of  geode
}

\NEW{Sign}{\sStorageLocker}{%%Flavor Text Sign!
  \s\MYname	{A Large Metal Storage Locker}
  \s\MYloc	{-004}
  \s\MYtext	{There is a large metal storage locker here. Inside is an old set of chipped china, a box of paperback books (they must be ancient!), a broken microscope, and a pair of galoshes with a hole in the left boot.}
  \s\MYitems {}
}

\NEW{Sign}{\sToolShelf}{%%Flavor Text Sign!
  \s\MYname	{A Long Shelf}
  \s\MYloc	{-004}
  \s\MYtext	{This shelf is covered in old tools. Some are so old that their uses are lost in time.
  
  {\bf If you would like to take an item from the shelf, place 1 hand on this sign for 30 seconds then take an item of your choice.}}
  \s\MYitems {\iZinc{} \iSyringe{} \iTungsten{} \iWireStripper{} \iPhotodiode{} \iCanOfBeans{}}
}

%%Pond | Water HQ
\NEW{Sign}{\sWaterPond}{%%Water elemental residue source
  \s\MYname	{A Still Pond of Water}
  \s\MYloc	{-032}
  \s\MYtext	{There is a pond of water here. You can't tell how deep it is in the middle, but here at the edge, it is shallow enough. The water is murky and full of algae; clearly no one has been taking care of it and there probably aren't any fish. 
  
  The pond can be used as a source of water.
  
  {\bf If you draw water from the pond for any reason, the item in which you put the water acquires an $\alpha$ score of ``2''.} (Write ``$\gamma$ = 2'' on the item.)}
  \s\MYitems {}
}

%% Meditative spot near Water HQ -- where avatar can make amaranth
\NEW{Sign}{\sMeditation}{
  \s\MYname     {A quiet spot}
  \s\MYloc      {-032}
  \s\MYtext     {This is a pleasant, quiet spot a little ways away from the pond and stream.  This would be a good place to meditate.  You cannot interact with this sign unless you know otherwise.}
  \s\MYitems {{\multi{6}{\iWhiteQ{}}}}
  %\s\MYitems {\iWhiteQ{} \iWhiteQ{} \iWhiteQ{} \iWhiteQ{} \iWhiteQ{} \iWhiteQ{}}
}

\NEW{Sign}{\sStream}{%%Flavor Text Sign
  \s\MYname	{A Small Stream}
  \s\MYloc	{-032}
  \s\MYtext	{A little stream trickles into the pond here. It is so overgrown with weeds and grasses that you almost didn't notice it.
  
  The steam can be used as a source of water.}
  \s\MYitems {}
}

\NEW{Sign}{\sWillow}{%%Flavor Text Sign
  \s\MYname	{A Willow Tree}
  \s\MYloc	{-032}
  \s\MYtext	{A willow tree hangs over the pond here, offering shade and a nice resting spot. What a great place to lie back and watch the clouds go by.}
  \s\MYitems {}
}

\NEW{Sign}{\sStables}{%%Flavor Text Sign
  \s\MYname	{An Empty Horse Stable}
  \s\MYloc	{-032}
  \s\MYtext	{There is an empty horse stable here. The family who lived here must have been incredibly rich to keep horses in the 27th century.}
  \s\MYitems {}
}


%%Kitchen Signs | Fire HQ
\NEW{Sign}{\sSink}{
  \s\MYname	{The Kitchen Sink}
  \s\MYloc	{-030}
  \s\MYtext	{This is a stainless steel kitchen sink.
  
  The sink can be used as a source of water.}
  \s\MYitems {}
}

\NEW{Sign}{\sRefrigerator}{
  \s\MYname	{An Industrial Sized Refrigerator}
  \s\MYloc	{-030}
  \s\MYtext	{This is a stainless steel refrigerator. It is used for storing food. 
  
  {\bf If you would like to take an item from the refrigerator or put one into it, place 1 hand on this sign for 30 seconds, then take an item of your choice out, or place 1 item inside.}}
  \s\MYitems {}
}

\NEW{Sign}{\sFreezer}{
  \s\MYname	{An Industrial Sized Freezer}
  \s\MYloc	{-030}
  \s\MYtext	{This is a stainless steel freezer. It is used for storing food and keeping things cold.
  
  {\bf If you would like to take an item from the freezer or put one into it, place 1 hand on this sign for 1 minute, then take an item of your choice out, or place 1 item inside.}}
  \s\MYitems {\iMintIceCream{} \iMintIceCream{} \iMintIceCream{}}
}

\NEW{Sign}{\sLavaPool}{
  \s\MYname	{A Lava Pool}
  \s\MYloc	{-030}
  \s\MYtext	{There is a lava pool sitting in the corner of the kitchen. It is unclear how it does not light the entire house on fire. It probably has something to do with the fire elementals. 
  
  {\bf If you would like to try to take an item from the Lava pool, place 1 hand on the sign for 30 seconds.}
    
  This lava pool can also be used as a heat source, {\bf but if you remain within 1 ZoC of the lava for more than 1 minute, you suffer from heat fatigue. Only fire elementals are immune to this effect.} (you can only walk heel to toe for the next 3 minutes).
  }
  \s\MYitems {\iLavaEmber{} \iLavaEmber{} \iLavaEmber{}}
}

\NEW{Sign}{\sKitchenCabinet}{
  \s\MYname	{A Kitchen Cabinet}
  \s\MYloc	{-030}
  \s\MYtext	{There is a kitchen cabinet against the wall here.
  
  {\bf If you would like to take an item from the cabinet or put one into it, place 1 hand on this sign for 30 seconds, then take an item of your choice out, or place 1 item inside.}
  }
  \s\MYitems {{\multi{3}{\iBowl{}}}}
}

\NEW{Sign}{\sFruitBowl}{
  \s\MYname	{A Fruit Bowl}
  \s\MYloc	{-030}
  \s\MYtext	{There is a bowl of fruit here.
  
  {\bf If you would like to take a fruit, place 1 hand on this sign for 30 seconds, then take a fruit of your choice.}
  }
  \s\MYitems {\iApple{} \iBanana{} \iTangerine{} \iPear{}}
}

\NEW{Sign}{\sWindow}{%%Flavor Text Sign
  \s\MYname	{A Bay Window}
  \s\MYloc	{-030}
  \s\MYtext	{You can look out over the grounds of the mansion from here.}
  \s\MYitems {}
}

\NEW{Sign}{\sTable}{%%Flavor Text Sign
  \s\MYname	{The Charred Remains of a Table}
  \s\MYloc	{-030}
  \s\MYtext	{The charred remains of a table are scattered across the floor. It must have been destroyed by a fire elemental.}
  \s\MYitems {}
}

%%Human HQ
\NEW{Sign}{\sGlassware}{
  \s\MYname	{A Glassware Cabinet}
  \s\MYloc	{-002}
  \s\MYtext	{This is a glass front cabinet for storing glassware. You can see many tiny vials inside the cabinet. I wonder why there are so many?
  
  {\bf If you would like to take a vial from the cabinet place 1 hand on this sign for 15 seconds, then take a vial.} }
  \s\MYitems {{\multi{10}{\iVial{}}}}
}

\NEW{Sign}{\sMap}{
  \s\MYname	{A Map of the World}
  \s\MYloc	{-002}
  \s\MYtext	{There is a map of the world on the wall. It shows the new boundaries of nations, and the areas decimated by elementals.}
  \s\MYitems {}
}

%%Hallway off of -002 | Tower (Air HQ)
\NEW{Sign}{\sPileOfShiny}{
  \s\MYname	{A Pile of Scrap Metal and Other Shiny Objects}
  \s\MYloc	{Hallway off -002}
  \s\MYtext	{There is a huge pile of scrap metal here. Air elementals must collect them for some reason. Maybe it is because they reflect the light of the sun? 
  
  If you would like to search the pile for something useful, {\bf place 1 hand on this sign for 30 seconds then take an item of your choice.}}
  \s\MYitems {{\multi{10}{\iZinc{}}} \iTungsten{} \iTungsten{} \iAluminum{} \iAluminum{}}
}

\NEW{Sign}{\sWindy}{%%AOE sign | Need to put a tape strip down on the floor to get attention
  \s\MYname	{A Windy Lookout}
  \s\MYloc	{Hallway off -002}
  \s\MYtext	{You can see the entire grounds from up here, but it is very windy!
  
  {\Huge {\bf It is so windy that unless you are an air elemental, you cannot spend more than 5 minutes here without going back to the main house for at least 3 minutes to warm up again.}}}
  \s\MYitems {}
}

\NEW{Sign}{\sDestroyedWall}{%%Flavor Text sign
  \s\MYname	{A Blown-Out Section of Wall}
  \s\MYloc	{Hallway off -002}
  \s\MYtext	{The air elementals must have blown this wall out to make it easier for them to get in and out of the tower.}
  \s\MYitems {}
}

%%Lab Ruins
\NEW{Sign}{\sPileOfTools}{
  \s\MYname	{A Pile of Discarded Tools}
  \s\MYloc	{Across from -004}
  \s\MYtext	{There is a pile of broken and discarded tools here. They must be left over from the original lab. Maybe some of them are still functional?
  
  If you would like to search the pile for something useful, {\bf place 1 hand on this sign for 30 seconds then take an item of your choice.}}
  \s\MYitems {{\multi{4}{\iSyringe{}}} \iWireStripper{} \iSiliconeMitt{}}
}

\NEW{Sign}{\sOrganizerBox}{
  \s\MYname	{A Hard Plastic Box For Organizing Electronics}
  \s\MYloc	{Across from -004}
  \s\MYtext	{There is a hard plastic box here. It is easily recognizable as an organizer for small electronics. It is miraculous that nothing fell on it when the lab was destroyed.
  
  If you would like to search the box for something useful, {\bf place 1 hand on this sign for 30 seconds then take an item of your choice.}}
  \s\MYitems {\iPhotodiode{} \iPhotodiode{}}
}

\NEW{Sign}{\sRubble}{%%Flavor Text Sign
  \s\MYname	{A Pile of Rubble}
  \s\MYloc	{Across from -004}
  \s\MYtext	{There is a pile of rubble here. You can't even tell what it used to be.}
  \s\MYitems {}
}

\NEW{Sign}{\sWall}{%%Flavor Text Sign
  \s\MYname	{A Crumbling Wall}
  \s\MYloc	{Across from -004}
  \s\MYtext	{This section of wall is crumbling alarmingly. It looks like the ceiling could come down at any minute.}
  \s\MYitems {}
}


%%Conduit location signs
%%need to assign to locations
\NEW{Sign}{\sLocationCoverSheet}{
  \s\MYname	{A Possible Location of Conduit Contact}
  \s\MYloc	{}%%TBD
  \s\MYtext	{It is possible that the conduit can be opened from this location. {\bf You must have an item capable of determining the ``Point of Conduit Contact'' in order to find out.}} 
}

\NEW{Sign}{\sSealedLocationSheet}{
  \s\MYname	{A Sealed Location}
  \s\MYloc	{}%%TBD
  \s\MYtext	{You cannot interact with this location until it is unsealed. ({\bf You cannot interact with this location unless you know otherwise.})} 
}

\NEW{SignMedium}{\sConduitOne}{
  \s\MYname	{A Possible Location of Conduit Contact}
  \s\MYloc	{}%%TBD
  \s\MYtext	{To determine if this is the current location of closest contact with the elemental plane, search the chart below for the number associated with the current time. The number associated with the current time is the {\bf only} active number. You may only check the number for the time slot corresponding to the \textbf{current or next half-hour slot}.
  
\begin{tabular}{ l || l }
  Time: & Active Number  \\ \hline \hline
  2:00-2:30 & 2922 \\ \hline
  2:30-3:00 & 7726 \\ \hline
  3:00-3:30 & 9826 \\ \hline
  3:30-4:00 & 1820 \\ \hline
  4:00-4:30 & 9291 \\ \hline
  4:30-5:00 & 1276 \\ \hline
  5:00-5:30 & 7865 \\ \hline
  5:30-6:00 & 9844 \\ \hline
\end{tabular}  } 
  }
  
\NEW{SignMedium}{\sConduitTwo}{
  \s\MYname	{A Possible Location of Conduit Contact}
  \s\MYloc	{}%%TBD
  \s\MYtext	{To determine if this is the current location of closest contact with the elemental plane, search the chart below for the number associated with the current time. The number associated with the current time is the {\bf only} active number. You may only check the number for the time slot corresponding to the \textbf{current or next half-hour slot}.
  
\begin{tabular}{ l || l }
  Time: & Active Number  \\ \hline \hline
  2:00-2:30  & 8766 \\ \hline
  2:30-3:00  & 1542 \\ \hline
  3:00-3:30  & 4098 \\ \hline
  3:30-4:00  & 1052 \\ \hline
  4:00-4:30  & 2242 \\ \hline
  4:30-5:00  & 3391 \\ \hline
  5:00-5:30  & 9994 \\ \hline
  5:30-6:00  & 9213 \\ \hline
\end{tabular}  } 
}

\NEW{SignMedium}{\sConduitThree}{
  \s\MYname	{A Possible Location of Conduit Contact}
  \s\MYloc	{}%%TBD
  \s\MYtext	{To determine if this is the current location of closest contact with the elemental plane, search the chart below for the number associated with the current time. The number associated with the current time is the {\bf only} active number. You may only check the number for the time slot corresponding to the \textbf{current or next half-hour slot}.
  
\begin{tabular}{ l || l }
  Time: & Active Number  \\ \hline \hline
  2:00-2:30  & 9130  \\ \hline
  2:30-3:00  & 6818  \\ \hline
  3:00-3:30  & 4380  \\ \hline
  3:30-4:00  & 4898  \\ \hline
  4:00-4:30  & 8786  \\ \hline
  4:30-5:00  & 6670  \\ \hline
  5:00-5:30  & 3814  \\ \hline
  5:30-6:00  & 7109  \\ \hline
\end{tabular}  
}   
}
  
\NEW{SignMedium}{\sConduitFour}{
  \s\MYname	{A Possible Location of Conduit Contact}
  \s\MYloc	{}%%TBD
  \s\MYtext	{To determine if this is the current location of closest contact with the elemental plane, search the chart below for the number associated with the current time. The number associated with the current time is the {\bf only} active number. You may only check the number for the time slot corresponding to the \textbf{current or next half-hour slot}.
  
\begin{tabular}{ l || l }
  Time: & Active Number  \\ \hline \hline
  2:00-2:30  & 6874  \\ \hline
  2:30-3:00  & 1948  \\ \hline
  3:00-3:30  & 3493  \\ \hline
  3:30-4:00  & 7420  \\ \hline
  4:00-4:30  & 3259  \\ \hline
  4:30-5:00  & 2757  \\ \hline
  5:00-5:30  & 3303  \\ \hline
  5:30-6:00  & 6724  \\ \hline
\end{tabular}
  }
}
  
\NEW{SignMedium}{\sConduitFive}{
  \s\MYname	{A Possible Location of Conduit Contact}
  \s\MYloc	{}%%TBD
  \s\MYtext	{To determine if this is the current location of closest contact with the elemental plane, search the chart below for the number associated with the current time. The number associated with the current time is the {\bf only} active number. You may only check the number for the time slot corresponding to the \textbf{current or next half-hour slot}.
  
\begin{tabular}{ l || l }
  Time: & Active Number  \\ \hline \hline
  2:00-2:30  & 9297  \\ \hline
  2:30-3:00  & 7034  \\ \hline
  3:00-3:30  & 5357  \\ \hline
  3:30-4:00  & 6598  \\ \hline
  4:00-4:30  & 5520  \\ \hline
  4:30-5:00  & 2218  \\ \hline
  5:00-5:30  & 3539  \\ \hline
  5:30-6:00  & 1833  \\ \hline
\end{tabular}  } 
  }
  
\NEW{SignMedium}{\sConduitSix}{
  \s\MYname	{A Possible Location of Conduit Contact}
  \s\MYloc	{}%%TBD
  \s\MYtext	{To determine if this is the current location of closest contact with the elemental plane, search the chart below for the number associated with the current time. The number associated with the current time is the {\bf only} active number. You may only check the number for the time slot corresponding to the \textbf{current or next half-hour slot}.
  
\begin{tabular}{ l || l }
  Time: & Active Number  \\ \hline \hline
  2:00-2:30  & 4819  \\ \hline
  2:30-3:00  & 8117  \\ \hline
  3:00-3:30  & 8279  \\ \hline
  3:30-4:00  & 2376  \\ \hline
  4:00-4:30  & 5570  \\ \hline
  4:30-5:00  & 6964  \\ \hline
  5:00-5:30  & 2923  \\ \hline
  5:30-6:00  & 4835  \\ \hline
\end{tabular}  } 
  }
  
\NEW{SignMedium}{\sConduitSeven}{
  \s\MYname	{A Possible Location of Conduit Contact}
  \s\MYloc	{}%%TBD
  \s\MYtext	{To determine if this is the current location of closest contact with the elemental plane, search the chart below for the number associated with the current time. The number associated with the current time is the {\bf only} active number. You may only check the number for the time slot corresponding to the \textbf{current or next half-hour slot}.
  
\begin{tabular}{ l || l }
  Time: & Active Number  \\ \hline \hline
  2:00-2:30  & 5709  \\ \hline
  2:30-3:00  & 4656  \\ \hline
  3:00-3:30  & 4114  \\ \hline
  3:30-4:00  & 9253  \\ \hline
  4:00-4:30  & 8538  \\ \hline
  4:30-5:00  & 1140  \\ \hline
  5:00-5:30  & 3916  \\ \hline
  5:30-6:00  & 5426  \\ \hline
\end{tabular}  } 
  }
  
\NEW{SignMedium}{\sConduitEight}{
  \s\MYname	{A Possible Location of Conduit Contact}
  \s\MYloc	{}%%TBD
  \s\MYtext	{To determine if this is the current location of closest contact with the elemental plane, search the chart below for the number associated with the current time. The number associated with the current time is the {\bf only} active number. You may only check the number for the time slot corresponding to the \textbf{current or next half-hour slot}.
  
\begin{tabular}{ l || l }
  Time: & Active Number  \\ \hline \hline
  2:00-2:30  & 2321  \\ \hline
  2:30-3:00  & 5399  \\ \hline
  3:00-3:30  & 5331  \\ \hline
  3:30-4:00  & 2435  \\ \hline
  4:00-4:30  & 4305  \\ \hline
  4:30-5:00  & 9863  \\ \hline
  5:00-5:30  & 8981  \\ \hline
  5:30-6:00  & 7745  \\ \hline
\end{tabular}  } 
  }
  
\NEW{SignMedium}{\sConduitNine}{
  \s\MYname	{A Possible Location of Conduit Contact}
  \s\MYloc	{}%%TBD
  \s\MYtext	{To determine if this is the current location of closest contact with the elemental plane, search the chart below for the number associated with the current time. The number associated with the current time is the {\bf only} active number. You may only check the number for the time slot corresponding to the \textbf{current or next half-hour slot}.
  
\begin{tabular}{ l || l }
  Time: & Active Number  \\ \hline \hline
  2:00-2:30  & 4042 \\ \hline
  2:30-3:00  & 7870 \\ \hline
  3:00-3:30  & 7982 \\ \hline
  3:30-4:00  & 2767 \\ \hline
  4:00-4:30  & 4383 \\ \hline
  4:30-5:00  & 6900 \\ \hline
  5:00-5:30  & 8732 \\ \hline
  5:30-6:00  & 3413 \\ \hline
\end{tabular}  } 
  }
  
\NEW{SignMedium}{\sConduitTen}{
  \s\MYname	{A Possible Location of Conduit Contact}
  \s\MYloc	{}%%TBD
  \s\MYtext	{To determine if this is the current location of closest contact with the elemental plane, search the chart below for the number associated with the current time. The number associated with the current time is the {\bf only} active number. You may only check the number for the time slot corresponding to the \textbf{current or next half-hour slot}.
  
\begin{tabular}{ l || l }
  Time: & Active Number  \\ \hline \hline
  2:00-2:30 & 7048   \\ \hline
  2:30-3:00 & 6015   \\ \hline
  3:00-3:30 & 4511   \\ \hline
  3:30-4:00 & 7884   \\ \hline
  4:00-4:30 & 7792   \\ \hline
  4:30-5:00 & 7423   \\ \hline
  5:00-5:30 & 9098   \\ \hline
  5:30-6:00 & 9824   \\ \hline
\end{tabular}  } 
  }
  
\NEW{SignMedium}{\sConduitEleven}{
  \s\MYname	{A Possible Location of Conduit Contact}
  \s\MYloc	{}%%TBD
  \s\MYtext	{To determine if this is the current location of closest contact with the elemental plane, search the chart below for the number associated with the current time. The number associated with the current time is the {\bf only} active number. You may only check the number for the time slot corresponding to the \textbf{current or next half-hour slot}.
  
\begin{tabular}{ l || l }
  Time: & Active Number  \\ \hline \hline
  2:00-2:30 & 8961  \\ \hline
  2:30-3:00 & 7920  \\ \hline
  3:00-3:30 & 5283  \\ \hline
  3:30-4:00 & 4591  \\ \hline
  4:00-4:30 & 9124  \\ \hline
  4:30-5:00 & 1461  \\ \hline
  5:00-5:30 & 9322  \\ \hline
  5:30-6:00 & 8739  \\ \hline
\end{tabular}  } 
  }
  
\NEW{SignMedium}{\sConduitTwelve}{
  \s\MYname	{A Possible Location of Conduit Contact}
  \s\MYloc	{}%%TBD
  \s\MYtext	{To determine if this is the current location of closest contact with the elemental plane, search the chart below for the number associated with the current time. The number associated with the current time is the {\bf only} active number. You may only check the number for the time slot corresponding to the \textbf{current or next half-hour slot}.
  
\begin{tabular}{ l || l }
  Time: & Active Number  \\ \hline \hline
  2:00-2:30 & 5789  \\ \hline
  2:30-3:00 & 9135  \\ \hline
  3:00-3:30 & 6844  \\ \hline
  3:30-4:00 & 3873  \\ \hline
  4:00-4:30 & 1449  \\ \hline
  4:30-5:00 & 6331  \\ \hline
  5:00-5:30 & 2444  \\ \hline
  5:30-6:00 & 5561  \\ \hline
\end{tabular}  } 
  }
  
\NEW{SignMedium}{\sConduitThirteen}{
  \s\MYname	{A Possible Location of Conduit Contact}
  \s\MYloc	{}%%TBD
  \s\MYtext	{To determine if this is the current location of closest contact with the elemental plane, search the chart below for the number associated with the current time. The number associated with the current time is the {\bf only} active number. You may only check the number for the time slot corresponding to the \textbf{current or next half-hour slot}.
  
\begin{tabular}{ l || l }
  Time: & Active Number  \\ \hline \hline
  2:00-2:30 & 3769 \\ \hline
  2:30-3:00 & 5375 \\ \hline
  3:00-3:30 & 4719 \\ \hline
  3:30-4:00 & 4818 \\ \hline
  4:00-4:30 & 8457 \\ \hline
  4:30-5:00 & 4944 \\ \hline
  5:00-5:30 & 7305 \\ \hline
  5:30-6:00 & 5488 \\ \hline
\end{tabular}  } 
  }
  
\NEW{SignMedium}{\sConduitFourteen}{
  \s\MYname	{A Possible Location of Conduit Contact}
  \s\MYloc	{}%%TBD
  \s\MYtext	{To determine if this is the current location of closest contact with the elemental plane, search the chart below for the number associated with the current time. The number associated with the current time is the {\bf only} active number. You may only check the number for the time slot corresponding to the \textbf{current or next half-hour slot}.
  
\begin{tabular}{ l || l }
  Time: & Active Number  \\ \hline \hline
  2:00-2:30 & 4359   \\ \hline
  2:30-3:00 & 9344   \\ \hline
  3:00-3:30 & 8695   \\ \hline
  3:30-4:00 & 3368   \\ \hline
  4:00-4:30 & 1819   \\ \hline
  4:30-5:00 & 5281   \\ \hline
  5:00-5:30 & 3896   \\ \hline
  5:30-6:00 & 2744   \\ \hline
\end{tabular}  } 
  }
  
\NEW{SignMedium}{\sConduitFifteen}{
  \s\MYname	{A Possible Location of Conduit Contact}
  \s\MYloc	{}%%TBD
  \s\MYtext	{To determine if this is the current location of closest contact with the elemental plane, search the chart below for the number associated with the current time. The number associated with the current time is the {\bf only} active number. You may only check the number for the time slot corresponding to the \textbf{current or next half-hour slot}.
  
\begin{tabular}{ l || l }
  Time: & Active Number  \\ \hline \hline
  2:00-2:30 & 6840 \\ \hline
  2:30-3:00 & 7955 \\ \hline
  3:00-3:30 & 2796 \\ \hline
  3:30-4:00 & 3711 \\ \hline
  4:00-4:30 & 1716 \\ \hline
  4:30-5:00 & 9637 \\ \hline
  5:00-5:30 & 4621 \\ \hline
  5:30-6:00 & 3409 \\ \hline
\end{tabular}  } 
  }
  
\NEW{SignMedium}{\sConduitSixteen}{
  \s\MYname	{A Possible Location of Conduit Contact}
  \s\MYloc	{}%%TBD
  \s\MYtext	{To determine if this is the current location of closest contact with the elemental plane, search the chart below for the number associated with the current time. The number associated with the current time is the {\bf only} active number. You may only check the number for the time slot corresponding to the \textbf{current or next half-hour slot}.
  
\begin{tabular}{ l || l }
  Time: & Active Number  \\ \hline \hline
  2:00-2:30 & 5931 \\ \hline
  2:30-3:00 & 6762 \\ \hline
  3:00-3:30 & 7524 \\ \hline
  3:30-4:00 & 9055 \\ \hline
  4:00-4:30 & 5842 \\ \hline
  4:30-5:00 & 7790 \\ \hline
  5:00-5:30 & 4730 \\ \hline
  5:30-6:00 & 9740 \\ \hline
\end{tabular}  } 
  }
  
\NEW{SignMedium}{\sConduitSeventeen}{
  \s\MYname	{A Possible Location of Conduit Contact}
  \s\MYloc	{}%%TBD
  \s\MYtext	{To determine if this is the current location of closest contact with the elemental plane, search the chart below for the number associated with the current time. The number associated with the current time is the {\bf only} active number. You may only check the number for the time slot corresponding to the \textbf{current or next half-hour slot}.
  
\begin{tabular}{ l || l }
  Time: & Active Number  \\ \hline \hline
  2:00-2:30 & 8691 \\ \hline
  2:30-3:00 & 7803 \\ \hline
  3:00-3:30 & 5059 \\ \hline
  3:30-4:00 & 9688 \\ \hline
  4:00-4:30 & 6520 \\ \hline
  4:30-5:00 & 8374 \\ \hline
  5:00-5:30 & 6365 \\ \hline
  5:30-6:00 & 3215 \\ \hline
\end{tabular}  } 
  }
  
\NEW{SignMedium}{\sConduitEighteen}{
  \s\MYname	{A Possible Location of Conduit Contact}
  \s\MYloc	{}%%TBD
  \s\MYtext	{To determine if this is the current location of closest contact with the elemental plane, search the chart below for the number associated with the current time. The number associated with the current time is the {\bf only} active number. You may only check the number for the time slot corresponding to the \textbf{current or next half-hour slot}.
  
\begin{tabular}{ l || l }
  Time: & Active Number  \\ \hline \hline
  2:00-2:30 & 1955  \\ \hline
  2:30-3:00 & 9108  \\ \hline
  3:00-3:30 & 9065  \\ \hline
  3:30-4:00 & 1361  \\ \hline
  4:00-4:30 & 4939  \\ \hline
  4:30-5:00 & 9179  \\ \hline
  5:00-5:30 & 6386  \\ \hline
  5:30-6:00 & 8829  \\ \hline
\end{tabular}   } 
  }