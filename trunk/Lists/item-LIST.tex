%%%%%
%%
%% This file creates the Item, ItemPacket, ItemFold, ItemEnvelope, and
%% ItemLabel datatypes, and creates macros for each.  These are for
%% various types of in-game items.
%%
%%%%%


%%%%%
%% Item macros are for normal item cards.
\DECLARESUBTYPE{Item}{TransElement}
\PRESETS{Item}{
  \FD\MYtext	{} %% longer text of item
  \FD\MYmark	{} %% possible contents of shaded ``mark'' on card
  \FD\MYbulky	{0} %% potential bulkiness
  \FD\MYcapacity{N/A} %% potential capacity
  \sd\MYlistmap	{\item\MYname\ifx\MYnumber\empty\else\ (\MYnumber)\fi}
  }


%%%%%
%% \prop
%% \unstash
%% \bulky{<number>}
%% \contain{<number>}
%%
%% \prop inside an Item macro labels the card as a prop.  \unstash
%% labels the card as unstashable.  \bulky{n} labels the card as
%% n-hands bulky.  \contain{n} labels the card with n-hands capacity.
\def\prop{%
  \append\MYmark{ ~PROP~ }}
\def\unstash{%
  \append\MYmark{ ~UNSTASHABLE~ }}
\def\bulky#1{%
  \s\MYbulky{#1}%
  \append\MYmark{\mbox{ ~\MYbulky-Hand~Bulky~ }}}
\def\contain#1{%
  \s\MYcapacity{#1}%
  \append\MYmark{\mbox{ ~\MYcapacity-Hand~Capacity~ }}}


%%%%%
%% ItemPacket macros are for item cards with an attached packet.
%% They are a subtype of Item.
\DECLARESUBTYPE{ItemPacket}{Item}
\PRESETS{ItemPacket}{
  \F\MYcontents
  }


%%%%%
%% ItemFold macros are for items represented by just a folded packet.
%% They are a subtype of ItemPacket, with the longer text and ``mark''
%% left blank, since they have no actual item card.
\DECLARESUBTYPE{ItemFold}{ItemPacket}
\PRESETS{ItemFold}{
  \s\MYmark{}
  }


%%%%%
%% ItemEnvelope macros are for items represented by just an envelope.
%% They are a subtype of ItemPacket, with the longer text and ``mark''
%% left blank, since they have no actual item card.
\DECLARESUBTYPE{ItemEnvelope}{ItemPacket}
\PRESETS{ItemEnvelope}{
  \s\MYmark{}
  }


%%%%%
%% ItemLabel macros are for small labels that would get used on
%% physreps, e.g. gun labels.  The ``mark'' is left blank, since
%% it isn't used for these.
\DECLARESUBTYPE{ItemLabel}{Item}
\PRESETS{ItemLabel}{
  \s\MYmark{}
  }


%%%%%
%% \icard[<extras>]{<name>}{<number>}{<text>}
%% \specialicard[<extras>]{<name>}{<number>}{<text>}{<mark>}
%% \itempacket[<extras>]{<name>}{<number>}{<text>}{<mark>}{<contents>}
%% \itemfold{<name>}{<number>}{<text>}{<contents>}
%% \itemenvelope{<name>}{<number>}{<text>}{<contents>}
%% \itemlabel{<name>}{<number>}{<text>}
%%
%% These are wrappers around \INSTANCE, useful for 1-shots.
%%
%% For \icard, \specialicard, and \itempacket, the optional <extras>
%% (in []'s) is for things like \unstash and \bulky{3}.  For example,
%% \icard[\prop\contain{2}]{..}{..}{..}{..} gives an item that has a
%% prop and 3-hands capacity.
%%
%% The last arg (#5) to \specialicard is for anything extra you may
%% want in the ``mark''
\newinstance{Item}{\icard[4][]}{
  \s\MYname{#2}\s\MYnumber{#3}\s\MYtext{#4}#1}
\newinstance{Item}{\specialicard[5][]}{
  \s\MYname{#2}\s\MYnumber{#3}\s\MYtext{#4}\s\MYmark{#5}#1}
\newinstance{ItemPacket}{\itempacket[6][]}{
  \s\MYname{#2}\s\MYnumber{#3}\s\MYtext{#4}\s\MYmark{#5}\s\MYcontents{#6}#1}
\newinstance{ItemFold}{\itemfold[4]}{
  \s\MYname{#1}\s\MYnumber{#2}\s\MYtext{#3}\s\MYcontents{#4}}
\newinstance{ItemEnvelope}{\itemenvelope[4]}{
  \s\MYname{#1}\s\MYnumber{#2}\s\MYtext{#3}\s\MYcontents{#4}}
\newinstance{ItemLabel}{\itemlabel[3]}{
  \s\MYname{#1}\s\MYnumber{#2}\s\MYtext{#3}}


%%%%%%%%%%%%%%%%%%%%%%%%%%%%%%%%%%%%%%%%%%%%%%%%%%%%%%%%%%%%%%%%%%

%\NEW{Item}{\iTest}{
%  \s\MYname	{Test Item}
%  \s\MYnumber	{0000}
%  \s\MYtext	{A Test Item Card}
%  }
%
%\NEW{ItemPacket}{\iTestPacket}{
%  \s\MYname	{Test Item}
%  \s\MYnumber	{0000}
%  \s\MYtext	{A Test Item with a big red button.  Open packet if
%		you press the big red button.}
%  \s\MYcontents	{The item beeps at you.}
%  }
%
%\NEW{ItemFold}{\iTestFold}{
%  \s\MYname	{Test Food}
%  \s\MYnumber	{0000}
%  \s\MYtext	{open if you eat}
%  \s\MYcontents	{It tastes yummy.}
%  }
%
%\NEW{ItemEnvelope}{\iTestEnvelope}{
%  \s\MYname	{Test Food}
%  \s\MYnumber	{0000}
%  \s\MYtext	{open if you eat}
%  \s\MYcontents	{It tastes yummy.}
%  }
%
%\NEW{ItemLabel}{\iTestLabel}{
%  \s\MYname	{Test Gun Label}
%  \s\MYnumber	{0000}
%  \s\MYtext	{Disc gun, loadable to 20 shots.}
%  }
%
%\NEW{Item}{\iWhatzit}{
%  \s\MYname	{Whatzit}
%  \s\MYnumber	{12345}
%  \s\MYtext	{If you press it, open packet a.  If you twirl it, open
%		packet b.  If you pull it, open packet c.}
%  \bulky	{1}
%  \s\MYsigns	{\signstrip{a}{it goes ``beep.''}
%		\signstrip{b}{it goes ``whoop.''}
%		\signstrip{c}{it goes ``bang.''}
%		}
%  \s\MYabils	{\ability{Stop Crying}{By futzing with the Whatzit, you
%		can make babies stop crying.}{I make the baby stop
%		crying.}
%		}
%  }

%%%%%%%%%%%%%%%%%%%%%%%%%%%%%%%%%%%%%%%%%%%%%%%%%%%%%%%%%%%%%%%%%%

\def\iWhiteNumber{543}
  
\NEW{ItemFold}{\iWhite}{
  \s\MYname	{White}
  \s\MYnumber	{\iWhiteNumber}
  \s\MYtext	{open if you eat}
  \s\MYcontents	{Whatever the effects of White are.}
  }

\NEW{ItemFold}{\iMintIceCream}{
  \s\MYname     {Mint Chocolate-Chip Ice Cream}
  \s\MYnumber   {}
  \s\MYtext     {open if you eat}
  \s\MYcontents {Mm, minty chocolately goodness.}
}

\NEW{Item}{\iElectromagnet}{
  %this is an important item but I forget how to signify it as unstashable.
  %this item would probably be best if physrepped
  \s\MYname     {A large electromagnet}
  \s\MYnumber   {145}
  \s\MYtext     {A bulky torus made of tightly-coiled wire.}
  \bulky{1}
  \unstash{} %%like such I believe
}

\NEW{Item}{\iWhitePowder}{%%HOW IS THIS DIFFERENT THAN THE "WHITE" ITEM ABOVE?
  %this is an important item but I forget how to signify it as unstashable.
  \s\MYname     {A mysterious white powder}
  \s\MYnumber   {\iWhiteNumber}
  \s\MYtext     {You do not know what this substance is.}
  \bulky{1} %%The white powder is 1 hand bulky? I though there was very little of it?!?
  \unstash{}
}

\NEW{Item}{\iTablet}{
  \s\MYname	{A stone tablet}
  \s\MYnumber	{101}
  \s\MYtext		{This is a heavy stone tablet, with writing inscribed into it.  It reads: \\
  To Dr. \cMS{\intro}:\\
  While you may not recall meeting me, I certainly remember encountering you and a friend of yours a few years ago.  Your friend was fantastically persuasive.  I think that kind of talent could be very useful.
  \\
  Now, while my initial involvement in your experiment was involuntary, and my big brother kind of interrupted you, I'm really interested in related technology that you're developing.  I'd love to help.  I might even keep quiet
  if you teach me about how to use it.  Thanks!
  \\
  \cMinion{}
  }
  \bulky{1}
}

%%Conduit Opening Items
\NEW{Item}{\iZinc}{
  \s\MYname	{A piece of Zinc}
  \s\MYnumber	{}
  \s\MYtext	{A strip of Zinc. It burns very brightly if lit on fire.}
}

\NEW{Item}{\iCopper}{
  \s\MYname	{A piece of Copper}
  \s\MYnumber	{}
  \s\MYtext	{A strip of copper. It has a few teeth marks on it. Has an earth elemental been chewing on it?}
}

\NEW{Item}{\iBattery}{
  \s\MYname	{A small, jury rigged battery}
  \s\MYnumber	{}
  \s\MYtext	{This battery can provide power to a machine for only a few minutes. Hopefully it will be enough.}
}

%%Tricorder, need to rename obviously
\NEW{Item}{\iTricorder}{
  \s\MYname	{Tricorder}
  \s\MYnumber	{}
  \s\MYtext	{A small, hand-held machine. This machine is designed to collect data on elementals in various ways. Refer to the associated research notebook for current functionality, and ways to research additional functionality.}
  %%\s\MYnotebooks{\nTricorder{}}
}

%%leader's anti enslavement shield. he can of course give it to whoever he chooses
\NEW{Item}{\iEleShield}{
  \s\MYname	{Prototype Anti-Enslavement Shield}
  \s\MYnumber	{}
  \s\MYtext	{This is a prototype of a machine to allow Cits to resist elemental enslavement. Once every 30 minutes, this machine can be activated to resist elemental enslavement for 2 minutes. During the cool-down period, the machine must also be reset before it can be used again. To reset the shield, a scientist must devote 1 minute to fixing it (role-play accordingly).}
}

%%Lab key door for MS lab. Dema also has a copy
\NEW{Item}{\iLabKey}{
  \s\MYname	{A key to the locked door}
  \s\MYnumber	{}
  \s\MYtext	{This key allows you to pass freely in and out of room -015.}
}
