%%%%%
%%
%% This file creates the Item, ItemPacket, ItemFold, ItemEnvelope, and
%% ItemLabel datatypes, and creates macros for each.  These are for
%% various types of in-game items.
%%
%%%%%


%%%%%
%% Item macros are for normal item cards.
\DECLARESUBTYPE{Item}{TransElement}
\PRESETS{Item}{
  \FD\MYtext	{} %% longer text of item
  \FD\MYmark	{} %% possible contents of shaded ``mark'' on card
  \FD\MYbulky	{0} %% potential bulkiness
  \FD\MYcapacity{N/A} %% potential capacity
  \sd\MYlistmap	{\item\MYname\ifx\MYnumber\empty\else\ (\MYnumber)\fi}
  }


%%%%%
%% \prop
%% \unstash
%% \bulky{<number>}
%% \contain{<number>}
%%
%% \prop inside an Item macro labels the card as a prop.  \unstash
%% labels the card as unstashable.  \bulky{n} labels the card as
%% n-hands bulky.  \contain{n} labels the card with n-hands capacity.
\def\prop{%
  \append\MYmark{ ~PROP~ }}
\def\unstash{%
  \append\MYmark{ ~UNSTASHABLE~ }}
\def\bulky#1{%
  \s\MYbulky{#1}%
  \append\MYmark{\mbox{ ~\MYbulky-Hand~Bulky~ }}}
\def\contain#1{%
  \s\MYcapacity{#1}%
  \append\MYmark{\mbox{ ~\MYcapacity-Hand~Capacity~ }}}
\def\physrepped#{%
  \append\MYmark{ ~PHYSREPPED~ }}


%%%%%
%% ItemPacket macros are for item cards with an attached packet.
%% They are a subtype of Item.
\DECLARESUBTYPE{ItemPacket}{Item}
\PRESETS{ItemPacket}{
  \F\MYcontents
  }


%%%%%
%% ItemFold macros are for items represented by just a folded packet.
%% They are a subtype of ItemPacket, with the longer text and ``mark''
%% left blank, since they have no actual item card.
\DECLARESUBTYPE{ItemFold}{ItemPacket}
\PRESETS{ItemFold}{
  \s\MYmark{}
  }


%%%%%
%% ItemEnvelope macros are for items represented by just an envelope.
%% They are a subtype of ItemPacket, with the longer text and ``mark''
%% left blank, since they have no actual item card.
\DECLARESUBTYPE{ItemEnvelope}{ItemPacket}
\PRESETS{ItemEnvelope}{
  \s\MYmark{}
  }


%%%%%
%% ItemLabel macros are for small labels that would get used on
%% physreps, e.g. gun labels.  The ``mark'' is left blank, since
%% it isn't used for these.
\DECLARESUBTYPE{ItemLabel}{Item}
\PRESETS{ItemLabel}{
  \s\MYmark{}
  }


%%%%%
%% \icard[<extras>]{<name>}{<number>}{<text>}
%% \specialicard[<extras>]{<name>}{<number>}{<text>}{<mark>}
%% \itempacket[<extras>]{<name>}{<number>}{<text>}{<mark>}{<contents>}
%% \itemfold{<name>}{<number>}{<text>}{<contents>}
%% \itemenvelope{<name>}{<number>}{<text>}{<contents>}
%% \itemlabel{<name>}{<number>}{<text>}
%%
%% These are wrappers around \INSTANCE, useful for 1-shots.
%%
%% For \icard, \specialicard, and \itempacket, the optional <extras>
%% (in []'s) is for things like \unstash and \bulky{3}.  For example,
%% \icard[\prop\contain{2}]{..}{..}{..}{..} gives an item that has a
%% prop and 3-hands capacity.
%%
%% The last arg (#5) to \specialicard is for anything extra you may
%% want in the ``mark''
\newinstance{Item}{\icard[4][]}{
  \s\MYname{#2}\s\MYnumber{#3}\s\MYtext{#4}#1}
\newinstance{Item}{\specialicard[5][]}{
  \s\MYname{#2}\s\MYnumber{#3}\s\MYtext{#4}\s\MYmark{#5}#1}
\newinstance{ItemPacket}{\itempacket[6][]}{
  \s\MYname{#2}\s\MYnumber{#3}\s\MYtext{#4}\s\MYmark{#5}\s\MYcontents{#6}#1}
\newinstance{ItemFold}{\itemfold[4]}{
  \s\MYname{#1}\s\MYnumber{#2}\s\MYtext{#3}\s\MYcontents{#4}}
\newinstance{ItemEnvelope}{\itemenvelope[4]}{
  \s\MYname{#1}\s\MYnumber{#2}\s\MYtext{#3}\s\MYcontents{#4}}
\newinstance{ItemLabel}{\itemlabel[3]}{
  \s\MYname{#1}\s\MYnumber{#2}\s\MYtext{#3}}


%%%%%%%%%%%%%%%%%%%%%%%%%%%%%%%%%%%%%%%%%%%%%%%%%%%%%%%%%%%%%%%%%%

%\NEW{Item}{\iTest}{
%  \s\MYname	{Test Item}
%  \s\MYnumber	{0000}
%  \s\MYtext	{A Test Item Card}
%  }
%
%\NEW{ItemPacket}{\iTestPacket}{
%  \s\MYname	{Test Item}
%  \s\MYnumber	{0000}
%  \s\MYtext	{A Test Item with a big red button.  Open packet if
%		you press the big red button.}
%  \s\MYcontents	{The item beeps at you.}
%  }
%
%\NEW{ItemFold}{\iTestFold}{
%  \s\MYname	{Test Food}
%  \s\MYnumber	{0000}
%  \s\MYtext	{open if you eat}
%  \s\MYcontents	{It tastes yummy.}
%  }
%
%\NEW{ItemEnvelope}{\iTestEnvelope}{
%  \s\MYname	{Test Food}
%  \s\MYnumber	{0000}
%  \s\MYtext	{open if you eat}
%  \s\MYcontents	{It tastes yummy.}
%  }
%
%\NEW{ItemLabel}{\iTestLabel}{
%  \s\MYname	{Test Gun Label}
%  \s\MYnumber	{0000}
%  \s\MYtext	{Disc gun, loadable to 20 shots.}
%  }
%
%\NEW{Item}{\iWhatzit}{
%  \s\MYname	{Whatzit}
%  \s\MYnumber	{12345}
%  \s\MYtext	{If you press it, open packet a.  If you twirl it, open
%		packet b.  If you pull it, open packet c.}
%  \bulky	{1}
%  \s\MYsigns	{\signstrip{a}{it goes ``beep.''}
%		\signstrip{b}{it goes ``whoop.''}
%		\signstrip{c}{it goes ``bang.''}
%		}
%  \s\MYabils	{\ability{Stop Crying}{By futzing with the Whatzit, you
%		can make babies stop crying.}{I make the baby stop
%		crying.}
%		}
%  }

%%%%%%%%%%%%%%%%%%%%%%%%%%%%%%%%%%%%%%%%%%%%%%%%%%%%%%%%%%%%%%%%%%

%%%-----------------------------%%%
%
% Items are organized by "`owned by a person at start game" and "somewhere in world"
% Withing "`person items"', organized by character
% Within "`somewhere in world"', organized by quest/mechanic
%
%%%-----------------------------%%%


\def\iWhiteNumber#{543}

%%Items that belong to people at game start:
  
\NEW{Item}{\iWhitePowder}{%%Azi Sci
  \s\MYname     {A mysterious white powder}
  \s\MYnumber   {\iWhiteNumber{}}
  \s\MYtext     {You do not know what this substance is.}
  \unstash{}
}

\NEW{Item}{\iElectromagnet}{%%GD
  \s\MYname     {A large electromagnet}
  \s\MYnumber   {145}
  \s\MYtext     {A bulky torus made of tightly-coiled wire.}
  \physrepped{}
  \bulky{1}
  \unstash{}
}

\NEW{Item}{\iTablet}{%%MS
  \s\MYname	{A stone tablet}
  \s\MYnumber	{101}
  \s\MYtext		{This is a heavy stone tablet, with writing inscribed into it.  It reads: \\
  To Dr. \cMS{\intro}:\\
  While you may not recall meeting me, I certainly remember encountering you and a friend of yours a few years ago.  Your friend was fantastically persuasive.  I think that kind of talent could be very useful.
  \\
  Now, while my initial involvement in your experiment was involuntary, and my big brother kind of interrupted you, I'm really interested in related technology that you're developing.  I'd love to help.  I might even keep quiet
  if you teach me about how to use it.  Thanks!
  \\
  \cMinion{}
  }
  \physrepped{}
  \bulky{1}
}

\NEW{Item}{\iAziTape}{%%MS
  \s\MYname	{Flash Drive with Azi Tape}
  \s\MYnumber	{}
  \s\MYtext	{A flash drive containing incomplete tape for rapid reeducation of Azi.  You never know when it might come in handy.}
  \unstash{}
  %\bulky{1}
}

%%Tricorder, need to rename obviously
\NEW{Item}{\iTricorder}{%%GD and MS
  \s\MYname	{Tricorder}
  \s\MYnumber	{}
  \s\MYtext	{A small, hand-held machine. This machine is designed to collect data on elementals in various ways. Refer to the associated research notebook for current functionality, and ways to research additional functionality.}
  \physrepped{}
}

%%leader's anti enslavement shield. he can of course give it to whoever he chooses
\NEW{Item}{\iEleShield}{%%Leader
  \s\MYname	{Prototype Anti-Enslavement Shield}
  \s\MYnumber	{}
  \s\MYtext	{This is a prototype of a machine to allow Cits to resist elemental enslavement. Once every 30 minutes, this machine can be activated to resist elemental enslavement for 2 minutes. During the cool-down period, the machine must also be reset before it can be used again. To reset the shield, a scientist must devote 1 minute to fixing it (role-play accordingly).}
  \unstash{}
}

%%Lab key door for MS lab. Dema also has a copy
\NEW{Item}{\iLabKey}{%%MS and Dema
  \s\MYname	{A Key}
  \s\MYnumber	{}
  \s\MYtext	{This key allows you to pass freely in and out of room -015.}
}

\NEW{ItemFold}{\iEleBelt}{%%Romeo
  \s\MYname	{A Non-Functional Elemental Protection Belt}
  \s\MYnumber	{}
  \s\MYtext	{When functioning, this belt is supposed to protect the wearer from Elemental Attacks. {\bf Do not open unless directed to do so.}}
  \s\MYcontents {{\bf The shield is now operational} and grants +3CR to defense against elementals who attack you. The item can be used by any human, although it provides \emph{no defense} against elemental enslavement.}
}

%%Elemental Essences
\NEW{Item}{\iFireEssence}{%%Fire Eles
  \s\MYname	{Fire Elemental Essence}
  \s\MYnumber	{}
  \s\MYtext	{{\large --BOUND--} This is a piece of fire elemental essence. 
  It is not in-game transferable  unless the owner choose to give it away, it is taken via a specific mechanic, or the owner dies - in which case the essence drops and becomes in-game transferable (in all cases, owner should cross out ``BOUND'').}
}

\NEW{Item}{\iEarthEssence}{%%Earth Eles
  \s\MYname	{Earth Elemental Essence}
  \s\MYnumber	{}
  \s\MYtext	{{\large --BOUND--} This is a piece of earth elemental essence. 
  It is not in-game transferable  unless the owner choose to give it away, it is taken via a specific mechanic, or the owner dies - in which case the essence drops and becomes in-game transferable (in all cases, owner should cross out ``BOUND'').}
}

\NEW{Item}{\iAirEssence}{%%Air Eles
  \s\MYname	{Air Elemental Essence}
  \s\MYnumber	{}
  \s\MYtext	{{\large --BOUND--} This is a piece of air elemental essence. 
  It is not in-game transferable  unless the owner choose to give it away, it is taken via a specific mechanic, or the owner dies - in which case the essence drops and becomes in-game transferable (in all cases, owner should cross out ``BOUND'').}
}

\NEW{Item}{\iQEssence}{%%Q Eles
  \s\MYname	{Quintessence Elemental Essence}
  \s\MYnumber	{}
  \s\MYtext	{{\large --BOUND--} This is a piece of quintessence elemental essence. 
  It is not in-game transferable  unless the owner choose to give it away, it is taken via a specific mechanic, or the owner dies - in which case the essence drops and becomes in-game transferable (in all cases, owner should cross out ``BOUND'').}
}


%%Items that start in game locations. Organized by mechanic/quest:
%%Avatar quest item
\NEW{ItemFold}{\iMintIceCream}{%%Freezer
  \s\MYname     {Mint Chocolate-Chip Ice Cream}
  \s\MYnumber   {}
  \s\MYtext     {Open if you eat.}
  \s\MYcontents {Mm, minty chocolately goodness.}
}

%%Battery Making mechanic
\NEW{ItemFold}{\iCopper}{%%Malachite
  \s\MYname	{A piece of Malachite}
  \s\MYnumber	{}
  \s\MYtext	{If you would like to extract pure copper from this item, hold it over a heat source (like lava or a fire elemental) for 10 seconds. Then open the item.}
  \s\MYcontents	{This item is now a {\bf strip of copper}. Copper makes a good cathode for batteries.}
}

\NEW{Item}{\iZinc}{%%pile of shiny
  \s\MYname	{A piece of Zinc}
  \s\MYnumber	{}
  \s\MYtext	{A strip of Zinc. It is highly reflective. Zinc makes a great anode for batteries.}
}

\NEW{Item}{\iBattery}{%%Chemical Stock Rooms (3)
  \s\MYname	{A small, jury rigged battery}
  \s\MYnumber	{}
  \s\MYtext	{This battery can provide power to a machine for only a few minutes. Hopefully it will be enough.}
}

%%Sealing the Conduit Locations Items
\NEW{Item}{\iVial}{%%Glassware cabinet in Human HQ
  \s\MYname	{A small glass vial}
  \s\MYnumber	{}%%Renumber later please
  \s\MYtext	{This vial is perfect for putting small volumes of liquids in.}
}

\NEW{Item}{\iSyringe}{%%Lab Ruins | Pile of TOols
  \s\MYname	{A Syringe}
  \s\MYnumber	{}%%Renumber later please
  \s\MYtext	{This syringe in sufficient for drawing blood from a human or elemental.}
}

%%Unsealing Conduit locations items
\NEW{Item}{\iBowl}{%%Kitchen Cabinet | Kitchen
  \s\MYname	{A Large Bowl}
  \s\MYnumber	{}
  \s\MYtext	{This is a large bowl. It is 1 hand bulky when empty, but {\bf 2 hands bulky when full}.}
  \bulky{1 or 2}
}

\NEW{Item}{\iLye}{%%JanitorCloset | Garden
  \s\MYname	{A Bottle of Lye}
  \s\MYnumber	{}
  \s\MYtext	{Lye is a caustic cleaning chemical. You should be careful when carrying it.}
  \bulky{1}
}

%%GD fetch items
\NEW{Item}{\iWireTightener}{%%Rogue, Janitors Closet
  \s\MYname	{A Wire Tightener}
  \s\MYnumber	{}
  \s\MYtext	{This is a wire tightener.}
}

\NEW{Item}{\iTungsten}{%%PileOfShiny | air HQ
  \s\MYname	{A Spool of Tungsten Wire}
  \s\MYnumber	{}
  \s\MYtext	{This is a small spool of shiny Tungsten Wire. It is fairly conductive but gets very hot when electricity is run through it.}
}

\NEW{Item}{\iWireStripper}{%%Pile of Tools | lab ruins
  \s\MYname	{A pair of Wire Strippers}
  \s\MYnumber	{}
  \s\MYtext	{These wire strippers can be used to strip insulation from a wire to facilitate connecting wires.}
}

\NEW{Item}{\iPhaseTransducer}{%%GD Lab Bench | GD Lab
  \s\MYname	{A Phase Transducer}
  \s\MYnumber	{}
  \s\MYtext	{This phase transducer is used to couple some technologies to the mesionic sensor grid. It must be rung whenever it is being used.}
  \physrepped{}
  \bulky{1}
}

%%AziSci quest items
\NEW{Item}{\iVanDeGraaff}{%%physically hidden | Lab Ruins
  \s\MYname	{A Van de Graaff generator}
  \s\MYnumber	{}
  \s\MYtext	{This generator is old but looks like it should still work.}
  \bulky{2}
}

\NEW{Item}{\iAluminum}{%%pile of shiny | Air HQ
  \s\MYname	{An Aluminum Rod}
  \s\MYnumber	{}
  \s\MYtext	{A light-weight aluminum rod.}
}

\NEW{Item}{\iPhotodiode}{%%organizing box | Lab Ruins
  \s\MYname	{A Photodiode}
  \s\MYnumber	{}
  \s\MYtext	{A small electronic piece designed to emit light.}
}

\NEW{ItemFold}{\iWhite}{%%Azi Sci
  \s\MYname	{Elementium}
  \s\MYnumber	{\iWhiteNumber{}}
  \s\MYtext	{Open if you eat}
  \s\MYcontents	{If you are an elemental: +1CR for 10 min. You feel extremely grateful toward the person who gave you this. You cannot attack or enslave this person for 10 minutes. {\bf If you are wounded, administering Elementium heals you instead}
  If you are human: Yuck, this powder has }
  \unstash{}
}

%%MS item fetch quest
\NEW{Item}{\iCoffeeMug}{%%CoffeeMachine | MS lab
  \s\MYname	{A Coffee Mug}
  \s\MYnumber	{}
  \s\MYtext	{This coffee mug is full of fresh-brewed coffee.}
  \bulky{1}
  \physrepped{}
}

\NEW{Item}{\iControlRod}{%%Control rod for mad scientist
  \s\MYname	{Prototype Control Rod}
  \s\MYnumber	{}
  \s\MYtext	{A large metallic rod with dials, flashing lights, and electrical wiring all over it.  It's not even clear how to hold it without getting electrocuted...but it's functionality and not aesthetics that counts, isn't it?  Besides, it's just a prototype.  There's plenty of time to make it pretty later.}
  \bulky{2}
}

%%Romeo Item Quest
\NEW{Item}{\iHeisenbergCompensator}{%%GD lab bench and Azi Sci lab bench | GD Lab, AziLab
  \s\MYname	{Heisenberg Compensator}
  \s\MYnumber	{}
  \s\MYtext	{This is a very advanced looking piece of technology.  You are impressed by the scientists who created it.}
  \bulky{1}
}

\NEW{ItemFold}{\iCrystalGeode}{%%Pile Of Rocks (sGeodes) | Cellar
  \s\MYname	{A Grey Rock}
  \s\MYnumber	{}
  \s\MYtext	{A heavy, hemispherical piece of rock; it is ugly, grey, and pitted. {\bf If you would like to break it open to check if it is a geode, hit it with a CR 4+ attack. If you do so, open the item}}
  \s\MYcontents {You break open the stone and see that {\bf it is indeed a Geode}. It has a polished interior, deep blue with brilliant crystals.}
  \bulky{1}
}

\NEW{ItemFold}{\iRocks}{%%Pile Of Rocks (sGeodes) | Cellar
  \s\MYname	{A Grey Rock}
  \s\MYnumber	{}
  \s\MYtext	{A heavy, hemispherical piece of rock; it is ugly, grey, and pitted. {\bf If you would like to break it open to check if it is a geode, hit it with a CR 4+ attack. If you do so, open the item.}}
  \s\MYcontents {You break open the stone and see that {\bf it is not a Geode}. It is just a dull, solid rock.}
  \bulky{1}
}

\NEW{Item}{\iLavaEmber}{%%Lava Pool | Kitchen
  \s\MYname	{Lava Ember}
  \s\MYnumber	{}
  \s\MYtext	{A palm-sized piece of stone that glows deep orange and feels warm to the touch.}
  \bulky{1}
}

\NEW{Item}{\iHolographicLaser}{%%Physically in lab space (no sign) | GD Lab
  \s\MYname	{Holographic Laser}
  \s\MYnumber	{}
  \s\MYtext	{A huge, fancy-looking piece of equipment that hulks in its corner.  It makes a whirring noise when it is turned on, and then shoots out a bright green beam.}
  \bulky{3}
}

%%Airplane Plot
\NEW{Item}{\iAirplanePiece}{%%Azi Lab Bench (can't carry it around) | Azi Lab
  \s\MYname	{Airplane Fragment}
  \s\MYnumber	{}
  \s\MYtext	{A twisted, scorched piece of metal from one of the airplanes that was crashed by an elemental.}
  \bulky{2}
}

%%Naturalist Research humans notebook
\NEW{ItemFold}{\iApple}{%%Fruit Basket | Kitchen
  \s\MYname	{A Red Apple}
  \s\MYnumber	{}
  \s\MYtext	{Open if you eat}
  \s\MYcontents{Yummy. You feel like you have lots of energy now.}
}

\NEW{ItemFold}{\iBanana}{%%Fruit Basket | Kitchen
  \s\MYname	{A Yellow Banana}
  \s\MYnumber	{}
  \s\MYtext	{Open if you eat}
  \s\MYcontents{Yummy. You feel like you have lots of energy now.}
}

\NEW{ItemFold}{\iTangerine}{%%Fruit Basket | Kitchen
  \s\MYname	{An Orange Tangerine}
  \s\MYnumber	{}
  \s\MYtext	{Open if you eat}
  \s\MYcontents{Yummy. You feel like you have lots of energy now.}
}

\NEW{ItemFold}{\iPear}{%%Fruit Basket | Kitchen
  \s\MYname	{A Green Pear}
  \s\MYnumber	{}
  \s\MYtext	{Open if you eat}
  \s\MYcontents{Yummy. You feel like you have lots of energy now.}
}

\NEW{ItemFold}{\iCanOfBeans}{%%ToolShelf | Kitchen
  \s\MYname	{A Can Of Beans}
  \s\MYnumber	{}
  \s\MYtext	{Open if you eat}
  \s\MYcontents{Yummy. You feel like you have lots of energy now.}
}

%%Flavor Text Items
\NEW{ItemFold}{\iWine}{%%Wine Rack | Cellar
  \s\MYname	{A Bottle of Wine}
  \s\MYnumber	{}
  \s\MYtext	{Open if you drink}
  \s\MYcontents{You feel tipsy for the next 5 minutes. Roleplay accordingly.}
}

\NEW{ItemFold}{\iMead}{%%Barrel | Cellar
  \s\MYname	{A Flagon of Mead}
  \s\MYnumber	{}
  \s\MYtext	{Open if you drink}
  \s\MYcontents{You feel tipsy for the next 5 minutes. Roleplay accordingly.}
}

