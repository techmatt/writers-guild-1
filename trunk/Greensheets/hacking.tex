\documentclass[green]{guildcamp1}
\begin{document}
\name{\gHacking{}}

\textbf{Electronic Lock Picking}

\emph{(This greensheet details how to use your computer skills to hack into any electronic lock. (note, this does not work on locks that require a physical key).)}

You are well versed in computer programming and thus have the skill to trick any electronic lock into opening for you.  This is a very delicate process that can be fairly time consuming. The difficulty of hacking the lock varies, depending on the lock itself. If anyone asks what you are doing, you must tell them that you are hooking up a small PDA to the lock which emits a series of beeps.

In general, if you wish to hack into a lock, you will interact with your deck of cards. The procedure for doing so is as follows:\\
1: If you are starting with a full deck, shuffle the deck 6 times\\
2: Deal out 5 cards in a line. This is your {\em working hand}.\\
3: Deal out 5 cards in a line below your working hand. This is your {\em dynamic library}.\\
4: You may swap out {\bf one} card in your {\em working hand} with one card in your {\em dynamic library}.\\
5: Discard all 5 cards in your {\em dynamic library}.\\
6: Repeat from 3 until you have fulfilled your success condition or you run out of deck.  If you run out of deck, you must start over from step 1.\\

{\bf Sucess Conditions:}
\begin{tabular}{||l|l||}
\hline\hline
Lock Difficulty	& Required hand:\\
\hline
1	& four-of-a-kind\\
2	& Royal Flush\\
3	& Royal Flush, THEN two pair\\
\hline\hline
\end{tabular}

\end{document}