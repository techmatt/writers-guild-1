\documentclass[green]{elementals}
\begin{document}
\name{\gConduitCover{}}

%%Cart goes Beep Beep

The machine is fairly complex and can do several things. It can be used to determine the point of closest contact between the human and elemental planes. It can also be used to open a conduit between the two planes.

NOTE: This machine is big, heavy and unwieldy. It is 3-hands bulky (one person may move it by walking heel to toe, two people may move it walking normally). If you move this machine, you must indicate that the machine is moving by saying ``Beep beep'' continuously. We wouldn't want to run someone over with it.\\

{\large How to Locate The Point of Closest Contact:}


This machine can locate where the Conduit can be opened. While it is known that the point of closest contact is in this general vicinity, the exact location remains unknown. To further complicate matters, the point of closest contact may move throughout the day as the planes approach each other.

In order to establish the point of closest contact (and thus the place where the conduit can currently be opened) you must follow these steps:
\begin{enumerate}
  \item Maneuver this machine to be directly in front of ``\sLocationCoverSheet{}'' Sign. ({\bf The sign may not read: ``\sSealedLocationSheet{}''.})
  \item Two people must put both hands on the machine for 30 seconds to indicate that you are using it to analyze the location.
  \item Once step 2 is complete, you may lift the ``\sLocationCoverSheet{}'' sign to check the sign under it. The sign will state when the location is active.
  \item When you are done reviewing the chart, you must cover it again with the ``\sLocationCoverSheet{}'' sign. If you need to check a location again, you must repeat this process from step 1, you can't just check the botttom page.
\end{enumerate}

\vspace{10mm}

{\large How to Open the Conduit:}

Read the entire process before beginning. A GM must be present before step 2 begins.

If you would like to open the conduit, you must follow these steps:
\begin{enumerate}
  \item Locate the current point of closest contact. (See above)
  \item Consume 1 energy source. (Tear up the item card)
  \begin{itemize}
  	\item \iBattery{} may be used if a human is participating in the opening.
  	\item \iCrystalGeode{} containing a blue geode may be used if an earth elemental is participating.
  	\item \iLavaEmber{} may be used if a fire elemental is participating.
  	\item \iHandCrank{} may be used if an air elemental is participating
  	\item you only need {\bf 1} energy source.
  \end{itemize}
  \item Declare which ability/abilities are being used for the opening. Each participant my contribute only one ability. Hold the ability card in your hand so it is visible for the duration of the ritual. At least 1 participant must use the ability ``\aWorkConduit{}''. 
  \item All players participating in the opening spend 3 minutes opening the conduit, roleplay accordingly.
  \item The GM will reveal what happens.
\end{enumerate}

\vspace{10mm}

{\large Important notes:}

\begin{itemize}
  \item Anyone who is not involved in opening the conduit or modifying the process in some way can walk up and observe the abilities being used. (\emph{The actions you are performing are obvious}).
  \item \emph{Withdrawing} abilities
  \begin{itemize}
  	\item If you participate in darkwater combat, notice a waylay attempt, or talk to a player not involved in opening the conduit, you \emph{withdraw} your action.
  	\item You may at any time choose to \emph{withdraw} your action and stop participating in the ritual. 
  	\item \emph{Withdrawn} abilities will not affect the outcome of the ritual.
  	\item ``\aWorkConduit{}'' is a special case. See ``Interrupting'' below.
  \end{itemize}
  \item \emph{Interrupting} the ritual
  \begin{itemize}
  	\item If the ``\aWorkConduit{}'' ability is withdrawn, the whole ritual is interrupted. It stops immediately and no effects are proccessed.
  	\item Someone {\bf not} participating in the ritual may interrupt the ritual by saying ``I stop you.''
  \end{itemize}
\end{itemize}

\end{document}
%%Machine has 1 number and instructions
%%Each conduit location has 2 pages. covered page has list of numbers that are active at different times
%% 1 for active, 0 for not active