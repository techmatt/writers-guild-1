%%%%%
%%
%% Research Notebooks live in this directory.  This file doubles as a
%% latex'able example notebook.
%%
%% Notebook macros (in ../Lists/notebook-LIST.tex, presumably) each
%% have a file that lives here.  The argument to \startnotebook{...}
%% probably should be the macro for the given whitesheet.  However,
%% you can also just use \name{Some Text} if you want.
%%
%% Note that every \startnotebook command needs a matching
%% \endnotebook command.  Also note that no ownership information
%% appears on the notebook.
%%
%%%%%

\documentclass[notebook]{elementals}
\begin{document}

%\item 
%\item 

\startnotebook{\nWhiteResearch{}}

\begin{page}{first}

This white powde (\iWhitePowder{}) you've found is very interesting; you need to do considerably more analysis. This would be so much easier if you were at a real lab instead of this hodge-podge of a ``mobile lab,'' but you'll make do. Your portable spectrometer is okay, but you remember that \cGD{} has brought a high-resolution mass spectrometer. Ask her if you can borrow it, and spend one minute getting her to help you use it on your sample.

When you complete this step, you may turn to \nbref{second}.

\end{page}

\begin{page}{second}

Hmm...you still don't know what this substance does, but it definitely contains particles involved in some very complicated parts of the elemental biological cycles. Use a \iTricorder{} to examine elementals from two different factions, spending at least one minute examining each.

When you complete this step, you may turn to \nbref{third}.

\end{page}

\begin{page}{third}

This powder is definitely integral to elemental biochemistry, but there are so many elementals here that it's hard to get a clear signal. To get a good baseline, ask \cGD{} if you can borrow her Chair of Science, then spend 30 seconds using a \iTricorder{} to examine two random humans.

When you complete this step, you may turn to \nbref{fourth}.

\end{page}

\begin{page}{fourth}

You've learned all you can from just examining the sample; time for some active testing. Convince an elemental to taste the powder while you examine them with your tricorder (this will destroy the item). Once they do so, hand them your ``W packet'' and tell them to open it.

When you complete this step, you may turn to \nbref{five}.

\end{page}

\begin{page}{five}

Amazing! The elemental's energy levels jumped dramatically upon consuming the powder, although there is so little of the substance that it is rapidly decaying back to its original level. You don't have a concrete hypothesis yet, but you believe this substance to be a necessary part of the elemental ``diet'' that is likely present on the elemental planes, but missing here on Earth. For ease of reference and because you are enamored with the name, you've started to refer to the substance as ``Elementium''.

To continue your experiments you will need to find a more stable sample of Elementium. Based on your research thus far, you believe you could make a device that would let you extract \emph{elemental essence} from an elemental. To do so, you will need to find an \iAluminum{} and a Van de Graaff generator (\iVanDeGraaff{\MYnumber}), then use these items on an elemental seated in the Chair of Science. At the end of this time, they must transfer an elemental essence to you, if they have any. (If you acquire elemental essence by some other means, you are welcome to use that instead).

Once you have elemental essence (from any faction of elementals) you can spend one minute using your Matter Transmuter to turn it into Elementium. \emph{Destroy the essence and take one Elementium item from the ``A'' packet on the ``\sTransmogrifier{}'' sign}.

When you have some Elementium, you may turn to \nbref{six}.

\end{page}

\begin{page}{six}

\emph{From now on, you may continue to use the Chair of Science to extract essence from elementals and then transmute elemental essence into Elementium.}

Excellent -- you now realize that your original sample of the substance had been decaying for many years, but this new sample of Elementium is both fresh and much larger. Unfortunately, for some reason it seems to be inert. Find another scientist and collaborate for three minutes while you discuss your experiences with Elementium to see if he or she has any insights into the problem. \emph{You must divulge what little you know about the substance, which is that it is important for several key stages of elemental biology.}

When you have finished collaborating, you may turn to \nbref{seven}.

\end{page}

\begin{page}{seven}

Ouch! Distracted by the discussion, as you were dividing the substance up you accidentally cut yourself on the scalpel. A few drops of your blood touched the sample. Initially, you were upset about the obvious contamination this caused, but you were soon shocked to see that the Elementium is no longer inert and is in fact now emitting dangerous amounts of energy (\emph{cross out the ``inert'' modifier on the item}). Time to gather more data!

\emph{You may now freely remove the ``inert'' tag on Elementium by using a drop of your blood. This does not require any items.}

Convince elementals from two different factions to try eating this newly activated Elementium while you scan them with a tricorder.

\emph{Eating Elementium consumes the item, so you will need to get another sample from somewhere. An elemental cannot eat Elementium you have transmuted from its own essence.}

When you have examined elementals from two different factions eating Elementium, you may turn to \nbref{eight}.

\end{page}

\begin{page}{eight}

You are very close to understanding the mysterious chemistry of Elementium, but you first need to conduct one more experiment: you need to expose the substance directly to the elemental plane. To accomplish this, you will need to work together with others to participate in a \emph{momentary opening} of the Conduit, no easy feat.

During an opening of the Conduit, you may put a sample of Elementium on the cart. The Elementium must be on the cart for the entire duration of the opening, while you periodically scan it and the surrounding area with your tricorder. Once the Conduit opening is completed, take the Elementium sample back to \cGD{}'s mass spectrometer for analysis, then turn to \nbref{nine}.

\end{page}

\begin{page}{nine}

Based on your analysis, you have come up with a very self-consistent theory regarding Elementium. It is a compound created by a fifth type of elemental which presumably live on the elemental plane. Without it, the elementals will die once their internal supply is depleted -- your back of the envelope calculation suggests the elementals on Earth have about 50 years before they will all die out, unless they find a source of Elementium. It also seems that elementals who consume this substance will eventually become dependent on whoever is supplying it to them: on the elemental plane, that must be this unknown fifth faction of elementals, but in your experiments today, that would be you.

You have not yet worked through all the political consequences, but Elementium is extremely abundant on the elemental plane. You think it might be possible to extract a \emph{huge} quantity of Elementium through the Conduit. However, you believe this Elementium will be inert, and are not sure there is any way to ``activate'' the substance without human blood. Your preliminary analysis shows the effects are similar to drug addiction in humans, and the effect will likely become more pronounced with time.

Inform \cDiplomat{} at once and discuss the ethical and political ramifications of this discovery. You also feel the need to discuss this with a few of the elementals you have been interacting with to see where they stand. If you want to try transporting large quantities of the substance through the Conduit, turn to \nbref{ten}.

%Gain social ability?

\end{page}

\begin{page}{ten}

\emph{Find a GM to gain the ``Conduit Modifier: Channel Elementium'' ability.}

All that remains is to calibrate the Conduit equipment so that the opening allows for the transportation of Elementium in addition to normal matter. To initiate the transfer, you will need to participate in a \textbf{terminal opening} of the Conduit. Place your ``Channel Elementium'' ability on the cart at the start of the opening. You must remain nearby to modulate the transfer until the opening is completed. You are not certain what form the substance will take once it comes through, but you expect it to be extremely dense.

If things go as planned, you should be able to transfer enough Elementium to sustain the elementals for several hundred years or more, although you are still not certain what the long-term consequences will be.

\end{page}

\endnotebook

\end{document}
