\documentclass[char]{elementals}
\begin{document}
\name{\cDema{}}

You were only fifteen when you killed your first elemental. 

A great beast of living fire, it filled the air with incandescent heat. But it would take more than a few flames to scare you. And, for all its power, the creature was predictable in its rage, and it was a simple matter to provoke it, lure it, and destroy it. As it hissed and steamed in the torrent of liquid nitrogen (courtesy of your mother's makeshift arms workshop) the men under your command took an involuntary step backward: fear, awe, even sorrow were clear in their expressions. But not you. As the once-mighty being writhed out its final moments, you allowed the faintest smile to touch your face. Even this primal being could die - and you were the one who had brought it down. \emph{This} was what true power felt like.

You are \cDema{\intro{}}. Your parents, and your parents' parents, and their parents before them, saw the elementals for what they were - an insatiable scourge, a formidable threat to humanity's continued ascendancy. You know the histories well: when the first beings came through the portal after \cGrandfather{\intro{}}'s failed experiment, most of humanity remembered the meaning of cowardice. They ran, they hid, they scattered like insects beneath a bootheel. But your ancestors were unbowed. They stood. They fought. And, yes, they died - but their fighting spirit burned on. Your mother and father were the proudest of your lineage, and they remembered.

As the leaders of the extremist faction known as the Extinguishers, they did what they could to stave off the elemental tide. You rose quickly through the ranks, gifted with the natural military talent and warrior's instincts of your family. At your hands, some of humanity's greatest victories were won - victories that humanity was far too quick to disown. 

Seven years ago, mankind's so-called \emph{leaders} - a mob of toadies, pacifists, and cowards - declared the Extinguishers a terrorist organization and, in what was meant as a ``conciliatory gesture'' to the same elementals who had ravaged your world, dedicated their own military resources to hunting them down. It was \emph{unthinkable} - these tyrants would not lift a finger to defend the Earth, but would callously destroy the very heroes who had sworn their lives to protecting it. They were nothing but scum, unworthy of the high offices they held.

Your parents - brave warriors, but burdened by their foolish ideals - chose to continue the struggle. You were more pragmatic: you knew that the Extinguishers' private army could not hope to stand against the military might of the \cNewUN{\intro{}}. The elementals still had the upper hand, and you were one of the few with the skills to fend them off. Without you, mankind didn't stand a chance. So you did what you had to do - you handed yourself over to the enemy. You denounced the Extinguishers, denounced the men and women who had fought at your side, denounced the very cause you had championed. And, in exchange, they gave you a meaningless political position and promptly forgot about you and your ``crimes.''

That was when you discovered something new: apparently the fiery passion that you had long used to inspire soldiers on the battlefield was just the sort of charisma a politician needed. You could use this talent. Your ardor (and tactical brilliance) was swiftly noted by those in positions of power, and you leapt from high office to high office until \cLeader{\intro{}} \cLeader{\themself}, the president of the \cNewUN{}, appointed you \cLeader{\their} secretary of state.

\cLeader{} was no different than any of the other politicians you had seen; in fact, \cLeader{\they} was among the worst. A snivelling idealist whose primary platform was compromise, \cLeader{\they} was foolish enough to believe that the elementals could be placated, or even bought off like a rival politician. But a position of power like this one was your best chance to reclaim the power that was yours by right, and perhaps to sabotage these hopeless sycophants along the way - so you accepted. 

You backed all but the most asinine of \cLeader{}'s policies, all the while claiming that \cLeader{\they} had shown you the light, that you were now on the side of peace. You didn't want to do anything \emph{too} drastic... not yet, at least. Two years ago was your hardest test: the task force that had been dispatched to deal with the Extinguishers finally captured your parents. During the farce of a trial, \cLeader{} admitted that they were to be ``made an example of'' - a summary execution for war crimes, to be carried out under the watchful eyes of your administration. The hardest thing you ever did was stand tall and hold back your tears as these \emph{traitors} sent your parents to their graves.

You knew then what you had suspected for more than a decade: none of these spineless cowards were fit to defend humanity. Perhaps with you in charge, things would run a bit more smoothly - no more traitors, no more elementals, just good old martial law. You'd show humanity how a real warrior ran things. You have not been idle since: there were plots to hatch, allies to make, people to deceive. 

To \cLeader{}'s credit, \cLeader{\they} is quite good at his job, despicable though \cLeader{\their} goals are. During your mutual administration, relations with the elementals became almost... friendly. At the very least, they seemed less inclined to raze a human city simply because it was ``in the way.'' It was during one of your many interminable negotiations with the elemental leaders that you realized something: the creatures hungered for power even more than you did. You still want to put them down like the beasts they are - but, before that time comes, they could be useful tools in your path to power.

It was a simple matter to schedule a meeting with the fire elementals, the most powerful of the four factions. Surviving that meeting was less simple (you still have the burn scar on your left calf from where their guards tried to threaten you), but you managed to convince \cQueen{\intro}, the leader of the fire elementals, that you meant business. You warned \cQueen{\them} that human technology presented an even greater threat to elementals (mostly due to your own reallocations of the President's R\&D budget, but you didn't bother to mention that part), and asked, in return, a small favor: her assistance in your ascent to power, once the time was right.

While \cQueen{} seemed unimpressed by your proposal, your words did have some impact. As you left the rendezvous, a solitary fire elemental approached you and your bodyguards. At first, you feared the thing was an assassin - but, once the creature introduced itself as \cPyro{\intro} and proposed an ``exchange of information,'' you realized that you had found a kindred spirit. It was all too easy to persuade \cPyro{\them} that your goals aligned - \cPyro{\they} would rule the elementals, you would rule the humans, and all would be well. Of course, you still planned to kill off the elementals the instant you seized power - but why make your new allies nervous?

As your web of connections grew, another useful figure came to your attention. An aspiring scientist and self-acknowledged genius known as \cMS{\intro}, \cMS{\they} privately claimed to be developing a rather unique technology that quickly caught your eye: a device that would grant the ability to \emph{enslave an elemental}. You knew that the elementals had the ability to turn even the strongest-willed human into an unthinking thrall - a power like this could turn the tide of any battle. Besides, having a few elemental minions under your \emph{personal} control would be the perfect way to ensure the success of your eventual coup d'etat. \cMS{\They}'s sort of a loose cannon, though - you trust \cMS{\them} about as far as you can throw \cMS{\them}. But, as long as \cMS{\they}'s useful...

And now, it seems, the perfect opportunity has arisen. \cLeader{} has managed to secure your embassy an invitation to the high council of the elemental leaders - conveniently located in \cGrandfather{}'s old laboratory. \cLeader{} seems convinced that \cLeader{\they}'ll be able to draft a bona fide treaty with one of the elemental factions - you're more concerned with the fact that all of your pawns will be in the same place at the same time. \cPyro{} will be accompanying the fire elementals, and \cMS{}, as one of humanity's foremost researchers, was asked to join the \cNewUN{}' entourage. In fact, you pulled a few strings yourself to get \cMS{\them} some \emph{secure} laboratory space, locked and sealed away from where prying eyes might uncover it (though you of course made sure a duplicate key was issued to you. You are, after all, not a moron). 

The elemental lords, and many of the humans at the conference, have their own inscrutable agendas. You couldn't care less what they want out of this gathering, of course, but you'll need to plan around your rivals if you want to stay on the path to power. And, as it turns out, the council is taking place in Strauss' burnt-out lab for a reason - the building houses the only conduit to the elementals' original plane of existence. The scientists in your staff have told you that the two planes are now in their closest convergence. This phenomenon has a variety of effects: it creates a sort of ``suppression field'' that holds the elementals' devastating powers in check, and it renders the portal itself physically tangible, even accessible by human means. You have no idea how the various factions might use the Conduit to their advantage, but, considering the might of the elementals, it's probably not an issue to take lightly.

On the other hand, it seems that you'd better keep \cLeader{} and \cLeader{\their} conciliatory fantasies from getting in the way of your plans as well. You're a military leader, not a bureaucrat, and the only way you'll stay in power (once you claim it, of course) is if there are battles to win - and the elementals are a convenient enemy to vilify. Come to think of it, a couple of weeks ago, your aides notified you of an unexpected plane collision over Hawai'i - a cataclysm that cost more than a hundred lives, and is already suspected to be the work of a rogue elemental. Perhaps, if you do some digging, you can use this to your advantage - though your contact \cPyro{} insists that fire, for once, had nothing to do with the incident.

If all goes well, by the end of today, you'll be back in your rightful place - at the head of humanity's armies, ruling and fighting with an iron fist and absolute power. 

\begin{itemz}[Goals]
	\item  Help \cMS{} perfect \cMS{\their} enslavement technology and test it out.
	\item  Prevent \cLeader{} and his diplomatic aides from making any lasting elemental-human treaties.
	\item  Find out more about the plane crash and pin it on one of the elemental factions.
	\item  Eliminate \cLeader{} or have \cLeader{\them} eliminated, and take over the Presidency yourself. Take care that no one catches you in the act - the blame should fall as far away from you as possible.
	\item  Make sure that no one uncovers or reveals any evidence of your scheming. Your plans are too critical to be undone by some bumbling fool.
\end{itemz}

\begin{contacts}
	\contact{\cLeader{}}  Your boss, the President: the most powerful man on the face of the earth.
	\contact{\cPyro{}}  Your contact among the fire elementals.
	\contact{\cMS{}}  A chaotic, possibly insane genius working on elemental enslavement.
	\contact{\cRomeo{}}  \cLeader{}'s Azi bodyguard. He's never trusted you, nor you him.
	\contact{\cQueen{}}  The ruler of the fire elementals. A terrifying creature.
	\contact{\cKing{}}  The ruler of the air elementals. \cLeader{} is currently negotiating with him.
  \contact{\cJuliet{}} A low-ranking fire elemental. You've encountered \cJuliet{\them} a few times during your negotiations.
\end{contacts}
You know there will be a few other diplomats and scientists at the conference, as well as hordes of elementals, but no one you know personally.

\end{document}
