\documentclass[char]{elementals}
\begin{document}
\name{\cMS{}}

You are \cMS{\intro}, a Cit and the best scientist humanity has. Barring some sect of underground academics who have managed to keep themselves hidden these past few decades, you are also the only person with the capability and vision to keep humanity from either being destroyed or enslaved by the elementals. People may call you mad, and yes, you do occasionally get caught up in your work, but as far as you know, you seem to be the only one sane enough to see what the elementals are doing. The politicians espouse a viewpoint of ``caution'' and ``balance'' and the elementals do indeed seem to be keeping each other in ``check'', but the equilibrium is so absurdly unstable you can't help but cackle a bit when you think about it.

It is plain as day what the motives of the elementals are: \cQueen{\intro}, the queen of the fire elementals, is infuriated she cannot get the upper hand on her rivals, and is at the Conduit because she believes she can bring through more of her kind to help wipe out the other elemental factions. Meanwhile, \cKing{\intro}, the king of the air elementals, believes the balance is already tipping in his faction's favor. He aims to leverage his friendship with the humans to further shift the battlefield while preventing \cQueen{} from bringing through allies of any sort. The king of the earth elementals is supposedly sick; but anyone of even modest intelligence can see how convenient this ``sickness'' is. Finally, the queen of the water elementals is quite aware that the elemental leaders here are practically certain to murder each other by day's end and is wisely staying as far away as possible. You may not know every little detail regarding elemental politics, but you don't have to because the general summary is clear as day: the elementals are at war and unless you do something, they will rip the world apart to kill each other and humanity will be crushed in the ensuing holocaust.

Unfortunately for the elementals, they are sorely underestimating humans. You know what they are up to, you know how they work, and you can use that knowledge to stop them. You have spent your entire career studying the elementals and are (by a considerable margin) the world's expert on the elementals' pseudo-psionic capabilities. To you, their capacity to enslave each other and humans is simultaneously a terrifying and thrilling power. It is an immensely complex process, but, it turns out, not one so complex you are incapable of understanding and controlling it. 

Sadly, your research is held back not by your intellect, but by the dangers of conducting experiments. To gather useful data, you require elemental test subjects. Towards this end, one person you have found very sympathetic to your cause is the secretary of state, \cDema{\intro}. However even with some off-the-books assistance from \cDema{}, you ultimately found that this type of research was labeled ``too dangerous'' or ``too politically unstable'' to pursue. Realizing how serious the situation is, you have continued to pursue your research in secret, but progress is painfully slow.

Your plan is to create a device capable of replicating the elementals' enslavement ability, mass produce it, and use it to tear the elemental factions apart from the inside. Unfortunately, you are far from the mass production stage -- you have only a single prototype (\iControlRod{}) and it is both constantly malfunctioning and, despite your ceaseless efforts, only able to enslave the weakest of elementals. A year ago, however, while studying exactly how Azi are able to largely resist the elemental's psionics, you came to a breakthrough that has revolutionized your work. Azi, it turns out, are not just resistant to enslavement but also a perfect conduit for replicating and controlling psionic powers. Empowered with this new insight, you spent months developing a new Azi tape (\iAziTape{}), the audio-visual stimuli used in their training. This tape enables the retaped Azi to mimic the enslavement power of even the most powerful elementals, rivaling even their kings and queens. You are confident that with enough retaped Azi working together, you could enslave one of the elemental leaders.

Illegally retaping Azi is a crime comparable to kidnapping or murder. Despite this, you felt it was absolutely necessary for you to test your tape on a real elemental. Pulling more than a few strings, you started by scheduling a ``day-long meeting'' with an Azi diplomat called \cDiplomat{\intro} at a relatively isolated lab near Earth elemental territory. Upon arriving at the lab, you promptly subdued and retaped \cDiplomat{}, then rapidly drove to the border of the Earth faction's territory to find a victim -- as \cDiplomat{}'s new handler, you had no trouble getting \cDiplomat{\them} to come along. After some searching, you managed to find a lone earth elemental loudly lumbering about whose power level was \emph{perfect} for your experiment. You could barely contain your excitement as you ordered \cDiplomat{} to enslave this unquiet wanderer. \cDiplomat{} rushed forward to do your bidding as you wielded your spectrograph from a safe distance.

Just thinking about how perfectly all your theorems and experiments came together to work just as you'd planned still brings a smile to your face, although it rapidly starts to fade as you recall what happened next. \cDiplomat{} was indeed easily able to control the weaker elemental. But the rather intense energy levels apparently emitted some form of high-frequency psionic wave that caught the attention of a nearby, considerably more powerful, elemental. Realizing your life was in danger, you were forced to rapidly recall \cDiplomat{} and retreat back to your lab, barely managing to escape the notice of the elder earth elemental. Despite your survival, you decided that this type of field work is simply not worth the risk: you will have to seek out expendable allies more suitable to the task. Once safely back in your lab, you managed to largely revert \cDiplomat{}'s memories and taping to its previous state. Annoyingly, \cScientist{\intro}, an Azi scientist and one of your most talented rivals, eventually detected the after effects of the retaping incident, although no evidence pointed back to you in the resulting investigation. You'll have to be careful around \cScientist{} today.

Acknowledging the need for someone better suited for the task of helping you test your inventions on elementals, you enlisted \cDema{}'s help and managed to set up a meeting with a high-ranking fire elemental called \cPyro{\intro} who you were confident was power-hungry enough to help you with your research. Things went well after your initial meeting, although it's quite clear you do not trust each other fully. You have met with \cPyro{} several times since then and \cPyro{\they} has been of some help. \cPyro{\They} could be a key component today.

Obviously, you were chosen to attend the upcoming meeting at the Conduit. There will be any manner of politicians present, and you are confident that as usual, they will manage to say exactly the wrong things at exactly the wrong time to make the situation explode. Of course, even without any humans present to complicate the situation, the elementals are also more than capable of wreaking havoc. Perhaps even more dangerous than the politicians attending are the other scientists. First, there is \cScientist{}, who specializes in elemental biochemistry, and whose ``ethical sensitivity'' is a constant pain; you suspect \cScientist{\they} has some sort of vendetta against you. Second, there is \cGD{\intro}, the great granddaughter of the long dead Konrad Strauss. Like her great grandfather, she appears to be doing extremely dangerous research into the nature of the Conduit itself. You do not know exactly what either of these scientists are up to, but you suspect \cScientist{} and especially \cGD{} have the potential to do something extremely reckless.

Overall the human contingent contains three Azi -- \cDiplomat{}, \cScientist{}, and the leader's bodyguard, \cRomeo{\intro}. Although \cDiplomat{}'s presence has the potential to complicate things a bit, overall you are keeping an eye on all three Azi, as you will likely need their aid (or at least, their bodies) if you start putting your plans into motion. Fortunately, \cDema{} has pulled some strings and gotten you your own private lab space at the Conduit site to conduct your experiments. \cDema{} may also have the authority to summon an azi secretary. This could be really convenient if it is too hard to retape the azi that are already here.

A few days ago you also received a missive on a stone tablet. Exhilarating and disturbing in equal measure, it appears to be from the earth elemental you enslaved some time ago, who identified \cMinion{\themself} as \cMinion{\intro}. You are not exactly thrilled that so many people seem to know what you are researching, but you do need more allies...especially elemental ones.

Despite the dangers, overall you are quite excited. The volatility of the situation at the Conduit site will afford many opportunities to make \emph{immense} progress in your work. If you play your cards right, you might even the chance to put your plans to gain control of the elemental factions into practice -- the ability to get close proximity to the faction leaders is not going to come again anytime soon.

\begin{itemz}[Notes]
\item You have stashed your \iControlRod{} for controlling elementals at your lab bench but can retrieve it whenever you feel it is safe to do so.
\end{itemz}

\begin{itemz}[Goals]
% thoughts: consider what the MS will think of the quintessence, once he hears about them. this may make more sense once we write the missive from Minion.
  \item Perfect your enslavement technology, and use it to control as many elementals as possible.
  \item Find and convince sympathetic humans that exploring and developing enslavement technology is the right thing to do.
  \item See if your Azi tape is powerful enough to enslave one of the leaders of an elemental faction, and if so use this control to either start a war between two factions or gain further control.
  \item Do not allow your lab to be searched, and do not let anyone find out that you retaped \cDiplomat{}.
\end{itemz}

\begin{contacts}
  \contact{\cDema{}} The secretary of state who seems interested in your research and occasionally helps you avoid politically sticky situations.
 \contact{\cPyro{}} A high-ranking fire elemental who sometimes helps you with your enslavement experiments and continues to express interest in your research.
	\contact{\cGD{}} A rival Cit scientist who specializes in the connection between Earth and the elemental plane.
	\contact{\cScientist{}} A rival Azi scientist who has in the past reported you for not following proper procedures to have your experiments approved.
	\contact{\cDiplomat{}} An Azi whom you retaped a year ago, although otherwise you are not very connected.
	\contact{\cMinion{}} An earth elemental whom you used in an important experiment. \cMinion{\They} recently sent you a very interesting letter.
\end{contacts}

\end{document}
