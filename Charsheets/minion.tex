\documentclass[char]{elementals}
\begin{document}
\name{\cMinion{}}

You are \cMinion{\intro}, a lowly earth elemental.

Your life was a happy one at first.  You broke off from your parent, and went happily about, merging with the earth, making earthquakes, swatting air elementals... you probably squished a couple of humans while you were at it, but they weren't important.

But the fun and games didn't last forever.  Everything started to slowly go wrong.  At first, you thought you had imagined the slow weakening of your strength.  Then, you were summoned into the presence of your king, \cEarthKing{}.  \cEarthKing{\They} had also experienced this loss of power, as had every other elemental in the room.  You were appropriately horrified.  \cEarthKing{} explained that it was likely that the lack of the beneficial influence of the quintessence elementals was to blame.  They had once been the fifth elemental, above all others in the hierarchy, enslaving all types of elementals and maintaining some kind of balance.  Since no quintessence had come through the Conduit long ago, the essential support they provided was slowly fading away.  And, without them, \cEarthKing{} thought it likely that all elementals would eventually weaken and die.

But, you, along with other elementals including \cLoyal{} and \cRogue{}, were assigned the task of resolving the problem.  Ten years ago, you visited the human structure where the Conduit had been opened before.  You snatched a convenient human along the way, to be a vessel for the incoming quintessence, the way humans had been used to bring in the other elementals a hundred years before.  Being controlled by the quintessence would be far superior to death.  With the others, you activated the human tools you thought would open the Conduit and bring a quintessence elemental through.

You broke some of the stuff.  Human instruments are so delicate!  But... something seemed to work.  Sort of.  There were wiggles on the screen you were watching, but you looked away at just the right moment, to see, for a single moment, a bright white glow around the human.  But then something broke.  The human collapsed, and there was no other elemental in the room.  All the others decided that they must have failed.  You wanted to stay and investigate further, but they made you leave with them.

You, on the other hand, are convinced that a quintessence elemental came through.  And, if the human you left behind actually survived the process, the quintessence elemental may be hiding out in that person.  You have explained this repeatedly to the other elementals, but without evidence, they don't believe you.  It's almost to the point where if you bring up the subject, they'll enslave you to make you shut up.  Now that you're returning to where the Conduit is, maybe you can find some actual evidence.  Or, even better, maybe the quintessence elemental will come back, trying to find a way home.  Unfortunately, you weren't paying as much attention to the humans then as you are now, and you probably won't recognize the human you nabbed back then.

That experiment ten years ago wasn't the only strange thing that's happened to you.  Three years ago, you were making some nice, comfy tremors when a human interrupted you.  And {\em enslaved} you, for a little while.  You were understandably shocked by this development.  You fled the human's control as soon as you could, helped by the appearance of an older and tougher earth elemental, and did a little skulking around afterwards to figure out what happened.  Apparently a human known as \cMS{\intro} was using another human to enslave you, but it didn't work as well as \cMS{\they} had hoped.  You decided to keep this quiet.  If something similar happens again, this information may suddenly become much more valuable.

At this elemental gathering near the Conduit, you and \cLoyal{} are supposed to be bring through enough quintessence elementals to establish order and keep all the elementals from dying.  But you've got a better idea.  Inspired by being on the receiving end of \cMS{}'s work, you think it would be best to implement \cLoyal{}'s plan, with a twist.  Instead of being enslaved by the incoming quintessence elementals, you will have some humans enslave them for you.  This will give the humans power, and it will support earth above the other elements, making them dependent on some kind of joint operation between earth and the humans for their survival.  You broached the subject with \cMS{} before the elemental gather, and \cMS{\they} seems supportive.  Especially since you offered to keep quiet in return for access to \cMS{\their} technology.

The long list of problems doesn't even stop there.  To top it all off, \cEarthKing{} has been poisoned.  Based on \cEarthKing{\their} reaction, \cEarthKing{\they} were dosed with a poison that used the essence drawn from a water elemental.  \cLoyal{} and \cRogue{} are both trying to keep this quiet, but it's going to get out eventually.  You don't know who was actually responsible, but if you can find some way to impugn the air elementals, that would be perfect.

\begin{itemz}[Goals]
	\item Find that quintessence elemental that came through ten years ago, or find evidence of it.  You're sure it has to be here, somewhere.
	\item Find a way to heal \cEarthKing{}.  \cMS{} may know something, or introduce you to other humans who might be good at this kind of thing.
	\item Find a way to enslave quintessence elementals, probably working with \cMS{}.
	\item Prevent the humans from forging an alliance with the air elementals.  And, ideally, convince them to work with you instead.  Offering them information is a good start.
	\item Convince \cLoyal{} that enslaving the quintessence is a good idea.  This might be easier after you've caught a quintessence elemental already.
\end{itemz}

\begin{contacts}
	\contact{\cLoyal{}}  The top-ranked elemental standing in for \cEarthKing{} at this gathering.
	\contact{\cRogue{}}  A second-tier elemental who has broken with the hierarchy.
	\contact{\cMS{}}  A human scientist studying elemental enslavement.
\end{contacts}

\end{document}
